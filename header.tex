%% Settings for single-side (simplex) printing
% Margins: left 40mm, right 25mm, top and bottom 25mm
% (but beware, LaTeX adds 1in implicitly)
\documentclass[12pt,a4paper]{report}
\setlength\textwidth{145mm}
\setlength\textheight{247mm}
\setlength\oddsidemargin{15mm}
\setlength\evensidemargin{15mm}
\setlength\topmargin{0mm}
\setlength\headsep{0mm}
\setlength\headheight{0mm}
% \openright makes the following text appear on a right-hand page
\let\openright=\clearpage

%% Settings for two-sided (duplex) printing
% \documentclass[12pt,a4paper,twoside,openright]{report}
% \setlength\textwidth{145mm}
% \setlength\textheight{247mm}
% \setlength\oddsidemargin{14.2mm}
% \setlength\evensidemargin{0mm}
% \setlength\topmargin{0mm}
% \setlength\headsep{0mm}
% \setlength\headheight{0mm}
% \let\openright=\cleardoublepage

%% Generate PDF/A-2u
\usepackage[a-2u]{pdfx}

%% Character encoding: usually latin2, cp1250 or utf8:
\usepackage[utf8]{inputenc}
% \usepackage[utf8]{fontspec}

%% Prefer Latin Modern fonts
\usepackage{lmodern}

%% Further useful packages (included in most LaTeX distributions)
\usepackage{amsmath}        % extensions for typesetting of math
\usepackage{amsfonts}       % math fonts
\usepackage{amsthm}         % theorems, definitions, etc.
\usepackage{bbding}         % various symbols (squares, asterisks, scissors, ...)
\usepackage{bm}             % boldface symbols (\bm)
\usepackage{graphicx}       % embedding of pictures
\usepackage{fancyvrb}       % improved verbatim environment
\usepackage{natbib}         % citation style AUTHOR (YEAR), or AUTHOR [NUMBER]
\usepackage[nottoc]{tocbibind} % makes sure that bibliography and the lists
			    % of figures/tables are included in the table
			    % of contents
\usepackage{dcolumn}        % improved alignment of table columns
\usepackage{booktabs}       % improved horizontal lines in tables
\usepackage{paralist}       % improved enumerate and itemize
\usepackage[usenames]{xcolor}  % typesetting in color

%%%%%%%%%%%%%%%%%%%%%%%%%%%%%%%%%%%%%%%%%%%%%%%%%%%%%%%%%%%%%%%%%%%%%
%%% User-defined packages
\usepackage{amssymb}
\usepackage[inline]{enumitem}
\usepackage{adjustbox}
\usepackage{dcolumn}
\usepackage{float}
\usepackage[section]{placeins}
\usepackage{morefloats}
\usepackage{mathtools}
% \usepackage{capt-of}
% \sloppy

% http://ftp.cvut.cz/tex-archive/macros/latex/contrib/algorithm2e/doc/algorithm2e.pdf
% https://en.wikibooks.org/wiki/LaTeX/Algorithms
\usepackage[]{algorithm2e}

% http://ftp.cvut.cz/tex-archive/macros/latex/contrib/acro/acro_en.pdf
% \acsetup{first-style=short}
\usepackage{acro}

\makeatletter
\renewcommand{\@algocf@capt@plain}{above}% formerly {bottom}
\makeatother

\renewcommand{\baselinestretch}{1.25}

\usepackage{caption}
\captionsetup[figure]{skip=10pt}

%%%%%%%%%%%%%%%%%%%%%%%%%%%%%%%%%%%%%%%%%%%%%%%%%%%%%%%%%%%%%%%%%%%%%
%%%%%%%%%%%%%%%%%%%%%%%%%%%%%%%%%%%%%%%%%%%%%%%%%%%%%%%%%%%%%%%%%%%%%

%%% Basic information on the thesis
% Thesis title in English (exactly as in the formal assignment)
\def\ThesisTitle{Evolutionary Algorithms for Data Transformation}

% Author of the thesis
\def\ThesisAuthor{Bc. Ondřej Švec}

% Year when the thesis is submitted
\def\YearSubmitted{2017}

% Name of the department or institute, where the work was officially assigned
% (according to the Organizational Structure of MFF UK in English,
% or a full name of a department outside MFF)
\def\Department{Department of Theoretical Computer Science and Mathematical Logic}

% Is it a department (katedra), or an institute (ústav)?
\def\DeptType{Department}

% Thesis supervisor: name, surname and titles
\def\Supervisor{Mgr. Martin Pilát, Ph.D.}
\def\SupervisorAvast{Ing. Martin Vejmelka, Ph.D.}

% Supervisor's department (again according to Organizational structure of MFF)
\def\SupervisorsDepartment{Department of Theoretical Computer Science and Mathematical Logic}

% Study programme and specialization
\def\StudyProgramme{Computer Science}
\def\StudyBranch{Artificial Intelligence}

% An optional dedication: you can thank whomever you wish (your supervisor,
% consultant, a person who lent the software, etc.)
\def\Dedication{%
I would like to thank my team at Avast Software, especially \SupervisorAvast, for inspiring me to investigate this area of machine learning and for his guidance from the very beginning of my research. Additionally, I would like to acknowledge the academic and technical support of Avast Software, namely for the provided hardware used for evaluating various deep learning methods. 

I would like to thank my supervisor, \Supervisor, for his guidance, the considerable time spent on consultations, proof reading, and for inspiring me to investigate the combination of metric learning and evolutionary algorithms.

I also want to thank my family for their continued support and encouragement during my Master studies and especially during the time spent working on this thesis.
}

% Abstract (recommended length around 80-200 words; this is not a copy of your thesis assignment!)
\def\Abstract{%
In this work, we propose a novel method for a supervised dimensionality reduction, which learns weights of a neural network using an evolutionary algorithm, CMA-ES, optimising the success rate of the $k$-NN classifier. If no activation functions are used in the neural network, the algorithm essentially performs a linear transformation, which can also be used inside of the Mahalanobis distance. Therefore our method can be considered to be a metric learning algorithm. By adding activations to the neural network, the algorithm can learn non-linear transformations as well. We consider reductions to low-dimensional spaces, which are useful for data visualisation, and demonstrate that the resulting projections provide better performance than other dimensionality reduction techniques and also that the visualisations provide better distinctions between the classes in the data thanks to the locality of the $k$-NN classifier.%
}

\def\AbstractCzech{%
V této práci jsme navrhli novou metodu pro supervised redukci dimenze, která se učí váhy neuronové sítě pomocí evolučního algoritmu CMA-ES, optimalizujícího úspěšnost $k$-NN klasifikátoru. Když v dané neuronové síti nejsou použity žádné aktivační funkce, tak algoritmus vykonává lineární transformaci. Tato lineární transformace také může být použita uvnitř Mahalanobisovy vzdálenosti a tím pádem naše metoda může být také považována za distance metric learning algoritmus. Při použití aktivačních funkcí v neuronových sítích se algoritmus může taky naučit nelineární transformace. V naší práci se zaměřujeme na redukci do nízko-dimenzionálních prostorů, které jsou užitečné pro vizualizaci dat. Experimentálně také ukazujeme, že ve srovnání s dalšími technikami pro redukci dimenze naše výsledné projekce fungují lépe a také ukazujeme, že naše vizualizace díky lokalitě $k$-NN klasifikátoru poskytují lepší interpretaci dat a rozlišení mezi různými třídami v datech.%
}

% 3 to 5 keywords (recommended), each enclosed in curly braces
\def\Keywords{%
{distance metric learning}, {Mahalanobis distance}, {dimensionality reduction}, {evolutionary algorithms}, {visualisation}, {data transformation}, {neural networks}
}

%% The hyperref package for clickable links in PDF and also for storing
%% metadata to PDF (including the table of contents).
% \usepackage[pdftex,unicode]{hyperref}   % Must follow all other packages
% % \usepackage[hidelinks]{hyperref}
% \hypersetup{breaklinks=true}
% \hypersetup{pdftitle={\ThesisTitle}}
% \hypersetup{pdfauthor={\ThesisAuthor}}
% \hypersetup{pdfkeywords=\Keywords}
% \hypersetup{urlcolor=blue}

\hypersetup{unicode}
\hypersetup{breaklinks=true}
