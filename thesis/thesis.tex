%%% The main file. It contains definitions of basic parameters and includes all other parts.

%% Settings for single-side (simplex) printing
% Margins: left 40mm, right 25mm, top and bottom 25mm
% (but beware, LaTeX adds 1in implicitly)
\documentclass[12pt,a4paper]{report}
\setlength\textwidth{145mm}
\setlength\textheight{247mm}
\setlength\oddsidemargin{15mm}
\setlength\evensidemargin{15mm}
\setlength\topmargin{0mm}
\setlength\headsep{0mm}
\setlength\headheight{0mm}
% \openright makes the following text appear on a right-hand page
\let\openright=\clearpage
\renewcommand{\baselinestretch}{1.2}

%% Settings for two-sided (duplex) printing
% \documentclass[12pt,a4paper,twoside,openright]{report}
% \setlength\textwidth{145mm}
% \setlength\textheight{247mm}
% \setlength\oddsidemargin{14.2mm}
% \setlength\evensidemargin{0mm}
% \setlength\topmargin{0mm}
% \setlength\headsep{0mm}
% \setlength\headheight{0mm}
% \let\openright=\cleardoublepage

%% Character encoding: usually latin2, cp1250 or utf8:
\usepackage[utf8]{inputenc}

%% Further useful packages (included in most LaTeX distributions)
\usepackage{amsmath}        % extensions for typesetting of math
\usepackage{amsfonts}       % math fonts
\usepackage{amsthm}         % theorems, definitions, etc.
\usepackage{bbding}         % various symbols (squares, asterisks, scissors, ...)
\usepackage{bm}             % boldface symbols (\bm)
\usepackage{graphicx}       % embedding of pictures
\usepackage{fancyvrb}       % improved verbatim environment
\usepackage{natbib}         % citation style AUTHOR (YEAR), or AUTHOR [NUMBER]
\usepackage[nottoc]{tocbibind} % makes sure that bibliography and the lists
			    % of figures/tables are included in the table
			    % of contents
\usepackage{dcolumn}        % improved alignment of table columns
\usepackage{booktabs}       % improved horizontal lines in tables
\usepackage{paralist}       % improved enumerate and itemize
\usepackage[usenames]{xcolor}  % typesetting in color

%%% User-defined packages
\usepackage{amssymb}
\usepackage[inline]{enumitem}
\DeclareMathOperator{\tr}{tr}
\usepackage{adjustbox}
\usepackage{dcolumn}
\usepackage{float}
\usepackage[section]{placeins}
% \usepackage{capt-of}

\usepackage{acro}
% http://tex.stackexchange.com/a/87223

\DeclareAcronym{psd}{
  short = PSD,
  long  = positive semi-definite,
  class = abbrev
}

\DeclareAcronym{pd}{
  short = PD,
  long  = positive definite,
  class = abbrev
}

\DeclareAcronym{knn}{
  short = kNN,
  long  = k-Nearest~Neighbour,
  class = abbrev
}

% http://ftp.cvut.cz/tex-archive/macros/latex/contrib/acro/acro_en.pdf
% \acsetup{first-style=short}


% define "struts", as suggested by Claudio Beccari in
%    a piece in TeX and TUG News, Vol. 2, 1993.
\newcommand\Tstrut{\rule{0pt}{2.6ex}}         % = `top' strut
\newcommand\Bstrut{\rule[-0.9ex]{0pt}{0pt}}   % = `bottom' strut


%%% Basic information on the thesis
% Thesis title in English (exactly as in the formal assignment)
\def\ThesisTitle{Evolutionary Algorithms for Data Transformation}

% Author of the thesis
\def\ThesisAuthor{Bc. Ondřej Švec}

% Year when the thesis is submitted
\def\YearSubmitted{2017}

% Name of the department or institute, where the work was officially assigned
% (according to the Organizational Structure of MFF UK in English,
% or a full name of a department outside MFF)
\def\Department{Department of Theoretical Computer Science and Mathematical Logic}

% Is it a department (katedra), or an institute (ústav)?
\def\DeptType{Department}

% Thesis supervisor: name, surname and titles
\def\Supervisor{Mgr. Martin Pilát, Ph.D.}
\def\SupervisorAvast{Ing. Martin Vejmelka, Ph.D.}

% Supervisor's department (again according to Organizational structure of MFF)
\def\SupervisorsDepartment{Department of Theoretical Computer Science and Mathematical Logic}

% Study programme and specialization
\def\StudyProgramme{Computer Science} %TODO CHECK THIS
\def\StudyBranch{Artificial Intelligence}

% An optional dedication: you can thank whomever you wish (your supervisor,
% consultant, a person who lent the software, etc.)
\def\Dedication{%
I would like to thank my team at Avast Software, especially \SupervisorAvast, for inspiring me to investigate this part of machine learning and his guidance from the very beginning of my research. Additionally, I would like to acknowledge the academic and technical support of the Avast Software, namely for the provided hardware used for evaluating various deep learning methods. 

I would like to thank my supervisor, \Supervisor, for his guidance, the considerable time spent on consultations, proof reading, and for inspiring me to investigate the combination of metric learning and evolutionary algorithms.

I also need to thank my family for their continued support and encouragement during my Master studies and especially during the time spent working on this thesis.
}

% Abstract (recommended length around 80-200 words; this is not a copy of your thesis assignment!)
\def\Abstract{%
Abstract. %TODO
}

% 3 to 5 keywords (recommended), each enclosed in curly braces
\def\Keywords{%
{Metric Learning} {Mahalanobis Distance} {Dimensionality Reduction} {Evolution Strategy} {Visualisation} %TODO
}

%% The hyperref package for clickable links in PDF and also for storing
%% metadata to PDF (including the table of contents).
\usepackage[pdftex,unicode]{hyperref}   % Must follow all other packages
\hypersetup{breaklinks=true}
\hypersetup{pdftitle={\ThesisTitle}}
\hypersetup{pdfauthor={\ThesisAuthor}}
\hypersetup{pdfkeywords=\Keywords}
\hypersetup{urlcolor=blue}

% Definitions of macros (see description inside)
%%% This file contains definitions of various useful macros and environments %%%
%%% Please add more macros here instead of cluttering other files with them. %%%

%%% Minor tweaks of style

% These macros employ a little dirty trick to convince LaTeX to typeset
% chapter headings sanely, without lots of empty space above them.
% Feel free to ignore.
\makeatletter
\def\@makechapterhead#1{
  {\parindent \z@ \raggedright \normalfont
   \Huge\bfseries \thechapter. #1
   \par\nobreak
   \vskip 20\p@
}}
\def\@makeschapterhead#1{
  {\parindent \z@ \raggedright \normalfont
   \Huge\bfseries #1
   \par\nobreak
   \vskip 20\p@
}}
\makeatother

% This macro defines a chapter, which is not numbered, but is included
% in the table of contents.
\def\chapwithtoc#1{
\chapter*{#1}
\addcontentsline{toc}{chapter}{#1}
}

% Draw black "slugs" whenever a line overflows, so that we can spot it easily.
\overfullrule=1mm

%%% Macros for definitions, theorems, claims, examples, ... (requires amsthm package)

\theoremstyle{plain}
\newtheorem{thm}{Theorem}
\newtheorem{lemma}[thm]{Lemma}
\newtheorem{claim}[thm]{Claim}

\theoremstyle{plain}
\newtheorem{defn}{Definition}

\theoremstyle{remark}
\newtheorem*{cor}{Corollary}
\newtheorem*{rem}{Remark}
\newtheorem*{example}{Example}

%%% An environment for proofs

%%% FIXME %%% \newenvironment{proof}{
%%% FIXME %%%   \par\medskip\noindent
%%% FIXME %%%   \textit{Proof}.
%%% FIXME %%% }{
%%% FIXME %%% \newline
%%% FIXME %%% \rightline{$\square$}  % or \SquareCastShadowBottomRight from bbding package
%%% FIXME %%% }

%%% An environment for typesetting of program code and input/output
%%% of programs. (Requires the fancyvrb package -- fancy verbatim.)

\DefineVerbatimEnvironment{code}{Verbatim}{fontsize=\small, frame=single}

%%% The field of all real and natural numbers
\newcommand{\R}{\mathbb{R}}
\newcommand{\N}{\mathbb{N}}

%%% Useful operators for statistics and probability
\DeclareMathOperator{\pr}{\textsf{P}}
\DeclareMathOperator{\E}{\textsf{E}\,}
\DeclareMathOperator{\var}{\textrm{var}}
\DeclareMathOperator{\sd}{\textrm{sd}}

%%% Transposition of a vector/matrix
\newcommand{\T}[1]{#1^\top}

%%% Various math goodies
\newcommand{\goto}{\rightarrow}
\newcommand{\gotop}{\stackrel{P}{\longrightarrow}}
\newcommand{\maon}[1]{o(n^{#1})}
\newcommand{\abs}[1]{\left|{#1}\right|}
\newcommand{\dint}{\int_0^\tau\!\!\int_0^\tau}
\newcommand{\isqr}[1]{\frac{1}{\sqrt{#1}}}

%%% Various table goodies
\newcommand{\pulrad}[1]{\raisebox{1.5ex}[0pt]{#1}}
\newcommand{\mc}[1]{\multicolumn{1}{c}{#1}}


%%% User-defined commands
% \newcommand{name}[num]{definition}
\newcommand{\cenfig}[3]{
\begin{figure}[ht]
    \centering
    \includegraphics[width=\textwidth,height=\textheight,keepaspectratio]{#1}
    \caption{#2} \label{#3}
\end{figure}
}


\newcommand*\rot{\rotatebox{90}}

\newcolumntype{L}[1]{>{\raggedright\let\newline\\\arraybackslash\hspace{0pt}}m{#1}}
\newcolumntype{C}[1]{>{\centering\let\newline\\\arraybackslash\hspace{0pt}}m{#1}}
\newcolumntype{R}[1]{>{\raggedleft\let\newline\\\arraybackslash\hspace{0pt}}m{#1}}

\DeclareMathOperator*{\argmin}{argmin}
\DeclareMathOperator*{\argmax}{argmax}
\DeclareMathOperator*{\tr}{tr}
\newcommand{\textapproxx}{\raisebox{0.5ex}{\texttildelow}}

\DeclarePairedDelimiter{\ceil}{\lceil}{\rceil}
\DeclarePairedDelimiter{\floor}{\lfloor}{\rfloor}

\newcommand{\twopartdef}[4]
{
	\left\{
		\begin{array}{ll}
			#1 & \mbox{if } #2 \\
			#3 & \mbox{if } #4
		\end{array}
	\right.
}

% define "struts", as suggested by Claudio Beccari in
%    a piece in TeX and TUG News, Vol. 2, 1993.
\newcommand\Tstrut{\rule{0pt}{2.6ex}}         % = `top' strut
\newcommand\Bstrut{\rule[-0.9ex]{0pt}{0pt}}   % = `bottom' strut

\newcommand\numberthis{\addtocounter{equation}{1}\tag{\theequation}}


% Title page and various mandatory informational pages
\begin{document}
\include{title}

%%% A page with automatically generated table of contents of the master thesis

\tableofcontents


%%%%%%%%%%%%%%%%%%%%%%%%%%%%%%%%%%%%%%%%%%%%%%%%%%%%%%%%%%%%%%%%%%%%%
%%%%%%%%%%%%%%%%%%%%%%%%%%%%%%%%%%%%%%%%%%%%%%%%%%%%%%%%%%%%%%%%%%%%%

% Guidelines

% Transformace dat jsou důležitou součástí strojového učení, která výrazně ovlivňuje kvalitu vytvořených modelů. Transformace dat se často používají i pro jejich zobrazení do prostoru s menší dimenzí, kde se dají snáze vizualizovat. Většina metod ale funguje bez učitele a tedy není schopna najít taková zobrazení dat, která by brala v úvahu vlastnosti metod strojového učení, které následují po ní, případně se snažila vizualizaci upravit tak, aby data ze stejné třídy byla blízko u sebe. Některé metody určené přímo pro vizualizaci (jako např. t-SNE) navíc ani neposkytují transformační funkci a nejsou tedy schopny zobrazit nová data bez přepočítání celého zobrazení. Cílem práce je tedy pomocí kombinace evolučních algoritmů a dalších přístupů (např. neuronových sítí) navrhnout metody pro vytvoření transformační funkce, která bude brát v úvahu i označkování dat. 

% Student se seznámí s postupy pro automatické transformace dat. Na základě zjištěných informací implementuje vlastní metody a porovná je s existujícími přístupy. Součástí srovnání bude i vhodnost použité metody pro zobrazení dat do prostoru s malou dimenzí vhodnou pro vizualizaci.

%%%%%%%%%%%%%%%%%%%%%%%%%%%%%%%%%%%%%%%%%%%%%%%%%%%%%%%%%%%%%%%%%%%%%
%%%%%%%%%%%%%%%%%%%%%%%%%%%%%%%%%%%%%%%%%%%%%%%%%%%%%%%%%%%%%%%%%%%%%

%\chapter*{Introduction}
\addcontentsline{toc}{chapter}{Introduction}

The need for measuring distance or similarity between data is ubiquitous in machine learning and creating custom metrics by hand is very difficult process. This has led to an introduction of distance metric learning where we learn distance metrics or similarities from data themselves.

One of the popular metrics is Mahalanobis distance, which generalizes Euclidean distance with a parameter matrix. We examine methods for learning this Mahalanobis distance and compare their performance on common datasets.

% Úvod do problematiky
% - Co už se udělalo
% - Co je známo
% - Co není známo
% - Co a pro Co a proč chceme naší prací objasnit naší prací objasnit
% - Jasná definice cíle práce
% - Struktura práce

% Related works
% - Co už udělali jiní
% - Co nového hodlá udělat autor

% v úvodu
% samostatná kapitola za úvodem samostatná kapitola za úvodem
% v závěru článku

%\chapter{Methods}

In this work we chose to examine the following methods:

\begin{enumerate}
\item Simple Covariance Metric
\item Xing
\item Large Margin Nearest Neighbor (LMNN)
\item Information Theoretic Metric Learning (ITML)
\item Sparse Determinant Metric Learning (SDML)
\item Least Squares Metric Learning (LSML)
\item Neighborhood Components Analysis (NCA)
\item Local Fisher Discriminant Analysis (LFDA)
\item Relative Components Analysis (RCA)
\end{enumerate}

\chapter{Experiments}

Datasets that we used for our experiments from 

https://archive.ics.uci.edu/ml/datasets/:

We chose the basic datasets that were mentioned in the original papers.

balance-scale, breast-cancer-wisconsin, ionosphere, iris, mice-protein, pima-indians-diabetes, sonar, soybean-large, wine

\chapter{Discussion}

Graphs, performace, caveats, what worked, what didnt work...

LMNN, ITML, SDML take too long to calculate

LMNN and LFDA perform very well

LMNN has too many hyper parameters

Simple Covariance metric was impossible to calculate on certain datasets

%\include{chap01}
%\include{chap02}
%\chapter{Mahalanobis metric}

\begin{align*}
  d(x,y)_{A} &= \sqrt{(x-y)^{T}A(x-y)} \\
         &= \sqrt{(x-y)^{T}L^{T}L(x-y)} \\
         &= \sqrt{(Lx-Ly)^{T}(Lx-Ly)} \\
         &= d(Lx, Ly)
\end{align*}
%\chapter*{Conclusion}
\addcontentsline{toc}{chapter}{Conclusion}

This worked, this didn't work. What we did.

\chapter{Introduction} \label{introduction}
\addcontentsline{toc}{chapter}{Introduction}

A metric is ubiquitous in machine learning. For instance, in classification, \ac{knn} classifier (@REF) uses a metric to find the nearest neighbours. In clustering, many clustering algorithms, including wildly used k-Means (@REF), calculate distance between data points and groups similar instances together. In information retrieval, when searching for the most relevant document, the documents themselves are ranked according to their similarity scores. Also security systems use a metric for face verification. 

It is obvious that performance of all these algorithms depends on the quality of the metric defined on the input space. The better the metric judges how close or how similar any two instances in the space are, the more the final machine algorithm will behave.

Several general purpose metrics exist, such as Euclidean distance, defined in equation \ref{eq:euclid}, or its generalized Minkowski distance, defined in equation \ref{eq:minkowski}, and cosine similarity, defined in equation \ref{eq:cosine} used for vectors of real numbers, or Levenshtein distance (also called edit distance) for measuring distances between strings.

% https://numerics.mathdotnet.com/distance.html
\begin{align}
d_{\mathbf{2}}(\textbf{x}, \textbf{y}) &= \|\textbf{x}-\textbf{y}\|_2 = \sqrt{\sum_{i=1}^{n} (x_i-y_i)^2} \label{eq:euclid} \\
d_{\mathbf{p}}(\textbf{x}, \textbf{y}) &= \|\textbf{x}-\textbf{y}\|_p = \bigg(\sum_{i=1}^{n} |x_i-y_i|^p\bigg)^\frac{1}{p} \label{eq:minkowski} \\
sim(\textbf{x}, \textbf{y}) &= \cos{\phi} = \frac{\textbf{x}\textbf{y}}{\|\textbf{x}\|\|\textbf{y}\|} \label{eq:cosine}
\end{align}

These metric, however, do not take any advantage of a prior knowledge of the task being solved or the structure of the data. Improved results are expected when the metric is designed for a particular task. Nevertheless, manually hand-crafting a metric is a strenuous and tedious process, which requires an expert knowledge of the task and the data. This has led to an introduction of distance metric learning where we learn distance metrics or similarities from data themselves.

Therefore the goal of metric learning is to adapt some pairwise metric function to the task at hand by using the training examples. This process requires almost no prior knowledge about the data and is very fast compared to hand-crafting a metric function, which may take months of experimenting. That is why metric learning algorithms that learn a distance metric or similarity directly from the data have been getting more and more traction in the recent years.

The origins of metric learning can be traced back to \cite{short1981optimal}, however the first paper that received a lot of attention was \cite{xing2002distance} where the metric learning is formulated as a convex optimization problem. Since then metric learning has been getting increasing amount of attention and appeared at conferences ICML 2010\footnote{\url{http://www.icml2010.org/program.html}}, ECCV 2010\footnote{\url{http://www.ics.forth.gr/eccv2010/tutorials.php}} and workshops at ICCV 2011\footnote{\url{http://www.iccv2011.org/authors/workshops}}, NIPS 2011\footnote{\url{https://nips.cc/Conferences/2011/Schedule}} and ICML 2013 \footnote{\url{http://icml.cc/2013/?page_id=41}}.

Another important area of machine learning is dimensionality reduction whose goal is to embed high-dimensional space in a low-dimensional space so that most of the information contained in the data is preserved. Some metric learning algorithms do not scale well when the data are high-dimensional and so the dimensionality reduction can be used to transform data into a low-dimensional space just to save computational time. Another dimensionality reduction application is visualisation where the data are reduced into a 2-dimensional or 3-dimensional space in which they can be plotted and inspected more easily.

% http://scikit-learn.org/stable/modules/decomposition.html#pca
% http://scikit-learn.org/stable/modules/lda_qda.html#id4
% http://scikit-learn.org/stable/modules/manifold.html
One of the most common unsupervised dimensionality reduction method is \ac{pca} \cite{jolliffe2002principal} which decomposes the dataset in a set of successive orthogonal components that explain a maximum amount of the variance inside of the data. Other well-known methods are for example \ac{ica} (@REF?) and \ac{lda} (@REF?). A specific area of non-linear dimensionality reduction is called manifold learning which is based on the idea that the dimensionality of many datasets is only artificially high but the intrinsic dimensionality is low. A popular manifold algorithm receiving a lot of traction recently is \ac{tsne} \cite{maaten2008visualizing} that tries to preserve a local structure of the data which is especially useful in visualisation. Other known manifold methods are \ac{lle} (@REF?), \ac{mds} (@REF?) and many others. What we found interesting and what we also focused on in this work is that many metric learning algorithms can be naturally used for dimensionality reduction.

In this thesis, we review existing state-of-the-art methods for learning global Mahalanobis distance metric (@LIST?) and we propose a new method using evolution algorithms ourselves. We compare all the methods on several popular datasets to see how they improve classification. Learning the full Mahalanobis metric means learning $d^2$ parameters and so we also try to restrict the existing methods to learn only a diagonal of the Mahalanobis distance metric with $d$ parameters. Additionally we explore the usage of metric learning algorithms for visualisation of the data (dimensionality reduction to 2 dimensions) and compare them with traditional approaches.

\section{Mahalanobis metric}
% http://blogs.sas.com/content/iml/2012/02/15/what-is-mahalanobis-distance.html
% https://en.wikipedia.org/wiki/Mahalanobis_distance

The problem with the Euclidean distance is that it does not account for variances in different directions and correlations between pairs of dimensions in the data. For normally distributed data, the distance from the mean can be specified by computing the z-score, defined as $z = \frac{x-\mu}{\sigma}$, where $\mu$ is the population mean and $\sigma$ is the population standard deviation. The z-score of $x$ can be looked at as a dimensionless quantity and can be interpreted as the number of standard deviations of $x$ away from the mean.

The figure \ref{fig:corrdata} demonstrates this problem in two dimensions: the figure visualises a dataset with both dimensions highly correlated and with the first dimension highly variable. In the plot the coloured ellipses show the probability density of the Gaussian mixture model. The probability density is high for ellipses near the origin and low for ellipses further away. In the figure there is $[0,0]$ (origin) and two other red points: $[0,1]$ and $[2,0]$. In Euclidean distance the point $[0,1]$ is twice closer to the origin compared to the point $[2,0]$. However, because the first axis has larger variance, it is clear that seeing point $[2,0]$ is much more probable in our dataset than it is seeing point $[0,1]$ because the point $[2,0]$ lies on the edge of the first ellipsis, meanwhile point $[0,1]$ lies beyond the edge of the second ellipsis. Using the terminology of the z-score, the point $[0,1]$ is more standard deviations away from the origin than the point $[2,0]$.

\cenfig{graphs/correlated_data}{Correlated data centered around origin}{fig:corrdata}

% http://blogs.sas.com/content/iml/2012/02/08/use-the-cholesky-transformation-to-correlate-and-uncorrelate-variables.html
The data can be uncorrelated using the covariance matrix $\bm{\Sigma}$, which is a \ac{psd} matrix and therefore the Cholesky decomposition $\bm{\Sigma}=\bm{L}\bm{L}^T$ can be used to obtain a lower triangular matrix $\bm{L}$. The inverse of this triangular matrix $\bm{L}$ can be used to remove the correlation from the data defined by $\bm{\Sigma}$. The figure \ref{fig:uncorrdata} demonstrates the effect of multiplying the original data by $\bm{L}^{-1}$. In this figure, it is now clear that the point at original coordinates $[2,0]$ is much closer to the origin in Euclidean distance than the the point with original coordinates $[0,1]$.

\cenfig{graphs/uncorrelated_data}{Uncorrelated data using Cholesky decomposition of covariance matrix}{fig:uncorrdata}

Many researches tackle metric learning by adapting a Mahalanobis distance metric first described in \cite{mahalanobis1936generalized} in \ref{eq:maha:orig}, where $\bm{\Sigma}$ is a covariance matrix of the data defined in equation \ref{eq:cov}. However this name is now overloaded and Mahalanobis distance now stands for any metric parametrized by a matrix $\bm{M}$ as defined in equation \ref{eq:maha}.

\begin{align} \label{eq:maha:orig}
d_{Mahalanobis}(\textbf{x},\textbf{y}) &= \sqrt{(\textbf{x}-\textbf{y})^T\bm{\Sigma}^{-1}(\textbf{x}-\textbf{y})}  \\
\bm{\Sigma} &= \mathrm {E} \left[\left(\mathbf {X} -\mathrm {E} [\mathbf {X} ]\right)\left(\mathbf {X} -\mathrm {E} [\mathbf {X} ]\right)^{\rm {T}}\right] \label{eq:cov}
\end{align}

\begin{equation} \label{eq:maha}
d_{\bm{M}}(\textbf{x},\textbf{y}) = \sqrt{(\textbf{x}-\textbf{y})^T\bm{M}(\textbf{x}-\textbf{y})} 
\end{equation}

The problem of correlation between dimensions does not arise in the Mahalanobis distance \eqref{eq:maha:orig} because it decorrelates the data intrinsically before calculating the distance itself. If the Cholesky decomposition is applied directly inside the equation \ref{eq:maha:orig}, we get that the Mahalanobis distance of the original data is equal to the Euclidean distance of the decorrelated data as shown in \ref{eq:mahalanobis:decorrelation}.

\begin{align}
  d_{Mahalanobis}(\textbf{x},\textbf{y}) &= \sqrt{(\textbf{x}-\textbf{y})^{T}\bm{\Sigma}^{-1}(\textbf{x}-\textbf{y})} \nonumber\\
         &= \sqrt{(\textbf{x}-\textbf{y})^{T}(\bm{L}\bm{L}^{T})^{-1}(\textbf{x}-\textbf{y})} \nonumber\\
         &= \sqrt{(\textbf{x}-\textbf{y})^{T}\bm{L}^{-T}\bm{L}^{-1}(\textbf{x}-\textbf{y})} \nonumber\\
         &= \sqrt{(\bm{L}^{-1}\textbf{x}-\bm{L}^{-1}\textbf{y})^{T}(\bm{L}^{-1}\textbf{x}-\bm{L}^{-1}\textbf{y})} \nonumber\\
         &= d_{\bm{2}}(\bm{L}^{-1}\textbf{x}, \bm{L}^{-1}\textbf{y}) \label{eq:mahalanobis:decorrelation}
\end{align}

Therefore from the \ref{eq:mahalanobis:decorrelation} it is possible to see the Mahalanobis distance has the following properties:
\begin{enumerate}
\item it automatically accounts for the scaling of the coordinate axes
\item it corrects for correlation between different features
\item it reduces to the Euclidean distance for uncorrelated data with unit variance
\end{enumerate}

The main weakness with the original Mahalanobis distance using the covariance matrix \ref{eq:maha:orig} is that it does not take the labels of the data instances into account (it can be thought of as "unsupervised" method). For this reason new metric learning algorithms exploiting additional information about data were developed and the term Mahalanobis distance now refers to not only decorrelation of the data using the Covariance matrix but any linear transformation using different matrices as defined in \ref{eq:maha}.

Not all matrices $\bm{M}$ in the generalized form of the Mahalanobis distance defined in equation \ref{eq:maha} would define a distance metric. For what matrices $\bm{M}$ does the equation remain a distance metric? From a mathematical perspective, a metric is a function $d(x,y)$ that defines a distance or dissimilarity between two instances from some input space: $d:\mathcal{X} \times \mathcal{X} \mapsto [0,\inf)$. For a function to be a metric it has to follow 4 conditions: nonnegativity, identity of indiscernibles, symmetry and triangle inequality as defined in equations \ref{eq:metricdef-1} --- \ref{eq:metricdef-4}.

% https://en.wikipedia.org/wiki/Metric_(mathematics)
\begin{align}
d(x,y) &\geq 0 & nonnegativity \label{eq:metricdef-1} \\
d(x,y) &= 0 \iff x=y & identity \ of \ indiscernibles \label{eq:metricdef-2} \\
d(x,y) &= d(y,x) & symmetry \label{eq:metricdef-3} \\
d(x,z) &\leq d_M(x,y) + d_M(y,z) & triangle \ inequality \label{eq:metricdef-4}
\end{align} 

In order for the Mahalanobis metric from equation \eqref{eq:maha} to be a metric, the matrix $\bf{M}$ that parametrizes the distance needs to pass the metric conditions \ref{eq:metricdef-1} --- \ref{eq:metricdef-4}. It turns out that a strictly \ac{pd} matrices $\bm{M} \succ 0$ do pass the conditions and therefore the Mahalanobis distance with a \ac{pd} matrix is a distance metric.

Sometimes, however, it is not easy to guarantee the positive definiteness of a matrix and we have to settle with a \ac{psd} matrix $\bf{M} \succeq 0$. For \ac{psd} matrices the identity of indiscernibles condition \ref{eq:metricdef-2} is not met and therefore the Mahalanobis distance with a \ac{psd} matrix is not a metric. For this reason the condition is usually relaxed into a condition of identity \ref{eq:pseudodef-2} which allows $d(x,y)=0$ for two distinct values $x \neq y$ whereas this is impossible using the identity of indiscernibles \ref{eq:pseudodef-2}. Together with the rest of the conditions, this forms a new set of conditions \ref{eq:pseudodef-1} ---\ref{eq:pseudodef-4}, which together defines a pseudo-metric. Therefore the Mahalanobis distance with a \ac{psd} matrix is a pseudo-metric.

%TODO fix corrent formatting for letters
\begin{align}
d(x,y) &\geq 0 & nonnegativity \label{eq:pseudodef-1} \\
d(x,x) &= 0 & identity \label{eq:pseudodef-2} \\
d(x,y) &= d(y,x) & symmetry \label{eq:pseudodef-3} \\
d(x,z) &\leq d(x,y) + d(y,z) & triangle \ inequality \label{eq:pseudodef-4}
\end{align} 

For the generalized Mahalanobis distance the same trick from the \ref{eq:mahalanobis:decorrelation} can be used for any general matrix $\bm{M}$ as can be seen in \ref{eq:mahalanobis:transform}. Cholesky decomposition is defined for any \ac{pd} matrix: $\bm{M}=\bm{L}\bm{L}^T$. Thus every Mahalanobis distance with a \ac{pd} matrix can be instead replaced by some linear transformation using a matrix $\bm{L}$ and simply calculating the Euclidean distance as usual.

\begin{align}
  d_{\bm{M}}(\textbf{x},\textbf{y}) &= \sqrt{(\textbf{x}-\textbf{y})^{T}\bm{M}(\textbf{x}-\textbf{y})} \nonumber\\
         &= \sqrt{(\textbf{x}-\textbf{y})^{T}\bm{L}^{T}\bm{L}(\textbf{x}-\textbf{y})} \nonumber\\
         &= \sqrt{(\bm{L}\textbf{x}-\bm{L}\textbf{y})^{T}(\bm{L}\textbf{x}-\bm{L}\textbf{y})} \nonumber\\
         &= d_{\bm{2}}(\bm{L}\textbf{x}, \bm{L}\textbf{y}) \label{eq:mahalanobis:transform}
\end{align}


%TODO WHY DO WE NEED OTHER METRIC LEARNING ALGORITHMS? (THINK OF AN EXAMPLE)

%TODO FIX ALL THIS
% Úvod do problematiky
% - Co už se udělalo
% - Co je známo
% - Co není známo
% - Co a pro Co a proč chceme naší prací objasnit naší prací objasnit MOTIVATION
% - Jasná definice cíle práce GOALS
% - Struktura práce

% Related works
% - Co už udělali jiní
% - Co nového hodlá udělat autor

%TODO HAVENT USED THIS YET
% curse of dimensionality

%%%%%%%%%%%%%%%%%%%%%%%%%%%%%%%%%%%%%%%%%%%%%%%%%%%%%%%%%%%%%%%%%%%%%

\section{Methodology}

\section{Applications}

Looking at metric learning as a "data transformation/preprocessing".

Computer vision
Information retrieval
Bioinformatics

\section{Related Topics}

Kernel learning
Multiple kernel learning
Dimensionality reduction

\section{Key Properties of Metric Learning Algorithms}

Learning paradigms: supervised, semi-supervised, unsupervised

Form of metrics: global (linear), non-linear, local

Scalability

Optimality of a solution (global optimum, local optimum)

Dimensionality reduction

%%%%%%%%%%%%%%%%%%%%%%%%%%%%%%%%%%%%%%%%%%%%%%%%%%%%%%%%%%%%%%%%%%%%%

\section{Evolution strategies} \label{chap:ea}
Quick intro

Describe EAs, ... crossover, mutation, selection (+schema)

\subsection{CMA-ES} \label{chap:ea:cmaes}
move to new chapter
\subsection{jDE} \label{chap:ea:jde}
move to related work


\section{Thesis structure}
%TODO

%%%%%%%%%%%%%%%%%%%%%%%%%%%%%%%%%%%%%%%%%%%%%%%%%%%%%%%%%%%%%%%%%%%%%

\chapter{Related work} \label{chap:rw}

[mention Bellet + make an outline + mention all possible metrics (online metrics)]

Metric learning has mostly been used to improve the performance of k-Nearest~Neighbour classification. Most of the research has been focused on learning this metric from labelled training instances, however researchers also focus on weakly-supervised tasks, where the labels of the instances are unknown, but the information about the data comes in the form of similar and dissimilar pairs:

\begin{align}
\mathcal{S} &= \lbrace(x_i,x_j): x_i \text{ and } x_j \text{ should be similar} \rbrace \\
\mathcal{D} &= \lbrace(x_i,x_j): x_i \text{ and } x_j \text{ should be dissimilar} \rbrace
\end{align}

Metric learning algorithm then tries to exploit these constraints to find the best parameters of a metric that best agrees with these constraints.

[Scalability in N and D] %TODO

In this work we chose to examine the following methods:

\begin{table}[ht] \centering
\caption{Summary of related metric learning methods} \label{tab:rw:summary}
\begin{tabular}{llllll}
\hline
% \multicolumn{4}{c}{Item} \\
% \cline{1-2}
Name & Year & Supervision & Optimum & Regularizer & Notes \\
\hline
MMC & 2002 & Weak & Global & None & — \\
LMNN & 2005 & Full & Global & None & For k-NN \\
NCA & 2004 & Full & Local & None & For k-NN \\
LFDA & 2007 & Full & Global & None & — \\
% RCA & 2003 & Weak & Global & None & — \\
% ITML & 2007 & Weak & Global & LogDet & Online version \\
% SDML & 2009 & Weak & Global & LogDet+L1 & $n \ll d$ \\
jDE.wFme & 2013 & Full & Local & None & Evolution \\
\hline
\end{tabular}
\end{table}

\cite{mahalanobis1936generalized}

\section{MMC Xing} \label{chap:rw:xing}
\cite{xing2002distance}

\section{Large Margin Nearest Neighbor (LMNN)} \label{chap:rw:lmnn}
\cite{weinberger2009distance}

% \section{Information Theoretic Metric Learning (ITML)} \label{chap:rw:itml}
% \cite{davis2007information}

% \section{Sparse Determinant Metric Learning (SDML)} \label{chap:rw:sdml}
% \cite{qi2009efficient}

% \section{Least Squares Metric Learning (LSML)} \label{chap:rw:lsml}
% \cite{liu2012metric}

% \section{Relative Components Analysis (RCA)} \label{chap:rw:rca}
% \cite{shental2002adjustment}

\section{Neighborhood Components Analysis (NCA)} \label{chap:rw:nca}
\cite{jacobgoldberger2004neighbourhood}

\section{Local Fisher Discriminant Analysis (LFDA)} \label{chap:rw:lfda}
\cite{sugiyama2007dimensionality}

\section{Evolutionary distance metric learning} \label{chap:rw:fukui}
\cite{fukui2013evolutionary}


- Teoretická analýza

Zmínit lokální metriky, online, ...

PCA is connected to Mahalanobis!!
% http://stats.stackexchange.com/questions/2691/making-sense-of-principal-component-analysis-eigenvectors-eigenvalues
% http://stats.stackexchange.com/questions/62092/bottom-to-top-explanation-of-the-mahalanobis-distance/62147#62147
->
formally explain that evolving 2*D matrix corresponds to some eigen value decomposition???

%%%%%%%%%%%%%%%%%%%%%%%%%%%%%%%%%%%%%%%%%%%%%%%%%%%%%%%%%%%%%%%%%%%%%

\chapter{Improving metric evolution} \label{chap:our-method}

main difference: we evolve L (no adjustments needed!)
proof that $L^TL$ is PSD matrix  (using SVD)
% http://www.deeplearningbook.org/contents/linear_algebra.html

CMAES + kNN

full matrix, diagonal matrix, neural network shape

k-Means for dimensionality reduction

\section{CMAES strategy} \label{chap:es:cmaes}

simple evolution, jDE, CMAES, ...

\begin{tabbing}
\hspace{50pt}\=\kill
$N$ \> number of samples \\
$D$ \> number of attributes \\ 
$G$ \> number of generations (200) \\
\end{tabbing} 

$\mathcal{O}(N^{2.376})$ matrix multiplication 

\begin{table}[ht] \centering
\begin{tabular}{rll}
\hline
Method & full matrix & diagonal matrix \\
\hline
Covariance & $\mathcal{O}(ND^2)$ & - \\
LMNN & $i*\mathcal{O}()$ & - \\
NCA & $i*\mathcal{O}(N^2D^2)$ & - \\
LFDA & $\mathcal{O}(N^2D)$ & - \\
CMAES.kNN & $\mathcal{O}()$ & - \\
CMAES.fMe & $\mathcal{O}()$ & - \\
jDE.kNN & $\mathcal{O}()$ & - \\
jde.fMe & $\mathcal{O}()$ & - \\ 
\hline
\end{tabular}
\caption{XXX} \label{tab:XXX}
\end{table}


jDE
$P=10*D^2$ population size
$\mathcal{O}(GP*(D^2+F_{eval}))$ for full
$\mathcal{O}(GP*(D+F_{eval}))$ for diagonal

CMAES
$P=4*3\log{D^2}$ population size
$\mathcal{O}(G*(PD+PD^2+PF_{eval}+D+D^2+D^3))$ for diagonal
% generate pop PD, PD^2
% update D, D^2
% eigen decomp D^3

Fitness
$N$ split
$D^2$ fit transformer (for full)
$ND^2$ transform samples (for full)

kNN
$N\log{N}$ build
$N^{1-\frac{1}{D}}+k$ retrieve

wFme ??

% http://stackoverflow.com/questions/9146086/time-complexity-of-genetic-algorithm
% Genetic Algorithms are not chaotic, they are stochastic. The complexity depends on the genetic operators, their implementation (which may have a very significant effect on overall complexity), the representation of the individuals and the population, and obviously on the fitness function. Given the usual choices (point mutation, one point crossover, roulette wheel selection) a Genetic Algorithms complexity is O(g(nm + nm + n)) with g the number of generations, n the population size and m the size of the individuals. Therefore the complexity is on the order of O(gnm)). This is of cause ignoring the fitness function, which depends on the application.

\section{k-Nearest~Neighbour fitness}

%%%%%%%%%%%%%%%%%%%%%%%%%%%%%%%%%%%%%%%%%%%%%%%%%%%%%%%%%%%%%%%%%%%%%

\chapter{Dimensionality Reduction} \label{chap:dim:reduction}

\section{Linear}

\section{Neural network transformation}


%%%%%%%%%%%%%%%%%%%%%%%%%%%%%%%%%%%%%%%%%%%%%%%%%%%%%%%%%%%%%%%%%%%%%

\chapter{Experiments and Examples}
We performed four different experiments to assess the performance of metric learning algorithms. In section \ref{chap:exp:classification} we use learnt metric in \ac{knn} classificator and compare classification errors on various datasets, which are described in \ref{chap:exp:datasets}. In the next experiment in section \ref{chap:exp:fitness} we compare different evolutionary algorithms together with various fitnesses and show how well they evolve and generalize. In section \ref{chap:exp:learning-times} we measure the training times of the algorithms and in the last experiment in section \ref{chap:exp:dimred} we use metric learning algorithms for visualisation and we compare the resulting embeddings.

\section{Experimental settings} 
In our experiments we compare metric learning algorithms listed in section \ref{chap:rw}. \ac{lmnn}, \ac{nca}, \ac{lfda} metric learning algorithms were already implemented in an open-sourced Python \textit{metric-learn} library \footnote{\url{https://github.com/all-umass/metric-learn}}. Only our evolutionary method described in section \ref{chap:our-method} and \cite{fukui2013evolutionary} were missing in this library and thus we designed a modular interface and implemented these two methods as a part of the \textit{metric-learn} library. The~implementation is described in chapter \ref{chap:impl}.

\subsection{Datasets} \label{chap:exp:datasets}
We chose to experiment on classification datasets that were most common among all related works, particularily in \cite{xing2002distance}, \cite{weinberger2009distance}, \cite{jacobgoldberger2004neighbourhood} and \cite{fukui2013evolutionary}. We found that the most common among all these papers were balance-scale, breast-cancer, iris, mice-protein, pima-indians, sonar and wine dataset. All these datasets were obtained from a well-known UCI Machine Learning Repository \footnote{\url{https://archive.ics.uci.edu/ml/datasets/}}. Even among these datasets mice-protein and sonar  have high dimensionality, however they are small in terms of number of samples. We also added digits6 and digits10 datasets, also obtained from the same archive, which are relatively larger datasets containing $8\times 8$ pixel images of digits with 6 and 10 classes respectively.

In order to test the metric learning algorithms on a highly variable data, we created an artificial dataset of four multinomial Gaussians. Each Gaussian has a different center, defined as a matrix in equation \eqref{eq:gauss:means} where each row corresponds to one center. All Gaussians were generated sharing one covariance matrix, defined in equation \eqref{eq:gauss:cov}. Dimensions are uncorrelated, however the first dimension has an enormous variability of $10^8$.

\begin{equation} \label{eq:gauss:means}
means = \begin{pmatrix}
10 & 0 & 0 & 2 \\
0 & 10 & 0 & -2 \\
0 & 0 & 10 & -2 \\
0 & 0 & -10 & 2 \\
\end{pmatrix}
\end{equation}
\begin{equation} \label{eq:gauss:cov}
covariance = \begin{pmatrix}
10^8 & 0 & 0 & 0 \\
0 & 100 & 0 & 0 \\
0 & 0 & 2 & 0 \\
0 & 0 & 0 & 1 \\
\end{pmatrix}
\end{equation}

Table \ref{tab:datasets} summarizes datasets that we experimented on. The table shows number of samples, dimensionality and number of distinct classes for every dataset. All the datasets are meant for classification task, meaning that all of their instances belong to some class. All the datasets have a labelled instances and so they can be used in a supervised methods. Most of the attributes of the datasets are real numbers, but some of them also contain integral values. The largest dataset in terms of number of samples and number of classes is digits10 dataset with 1797 samples. However the largest dataset in terms of number of dimensions is mice-protein dataset. Most of the datasets are relatively low in both number of samples, which is under a thousand, and their dimensionality is usually under 10.

\begin{table}[ht] \centering
\begin{tabular}{lrrr}
\hline
% \multicolumn{4}{c}{Item} \\
% \cline{1-2}
Dataset & \#~samples & \#~dimensions & \#~classes \\
\hline
balance-scale           & 625   & 4    & 3  \\
breast-cancer           & 699   & 9    & 2  \\
digits6                 & 1083  & 64   & 6  \\
digits10                & \textbf{1797}  & 64  & \textbf{10} \\
gaussians               & 400   & 5   & 4  \\
iris                    & 150   & 4    & 3  \\
mice-protein            & 1080  & \textbf{77}   & 8  \\
pima-indians            & 768   & 8    & 2  \\
sonar                   & 208   & 60   & 2  \\
wine                    & 178   & 13   & 3  \\
%soybean-large           & 307   & 35   & \textbf{19} \\
%ionosphere              & 351   & 34   & 2  \\
%letters                 & 20000 & 16   & 26 \\
%mnist                   & 70000 & 784  & 10 \\
%ofaces                  & 400   & 4096 & 40 \\
\hline
\end{tabular}
\caption{Datasets overall summary} \label{tab:datasets}
\end{table}

\subsection{Preprocessing data} \label{chap:exp:preprocessing}
Some of the datasets have missing attributes. There are several strategies for dealing with missing attributes, such as removing affected samples, filling missing values with a random value or zeros, their mean, median or the most frequent value. We chose to fill missing values using the mean for any given attribute as we found it behaves well with all the chosen datasets.

\begin{table}[ht] \centering
\begin{tabular}{lrrrrr}
\hline
Dataset & minimum & maximum & mean & std. deviation \\
\hline
balance-scale           & 1.00  & 5.00    & 3.00  & 1.41 \\
breast-cancer           & 1.00  & 10.00   & 3.13  & 2.88 \\
digits6                 & 0.00  & 16.00   & 4.87  & 6.04 \\
digits10                & 0.00  & 16.00   & 4.88  & 6.02 \\
gaussians               & \textbf{-33870.07} & \textbf{31033.73} & -7.65 & \textbf{5285.16} \\
iris                    & 0.10  & 7.90    & 3.46  & 1.97 \\
mice-protein            & -0.06 & 8.48    & 0.68  & 0.79 \\
pima-indians            & 0.00  & 846.00  & 44.99 & 58.37 \\
sonar                   & 0.00  & 1.00    & 0.28  & 0.28 \\
wine                    & 0.13  & 1680.00 & \textbf{69.13} & 215.75 \\
\hline
\end{tabular}
\caption{Datasets samples statistics} \label{tab:datasets-samples}
\end{table}

Another difficulty with the datasets is that many of them do not have normalized attributes and their attribute values are unbounded and highly variable. From table \ref{tab:datasets-samples} it is clear that the most variable dataset is our artificial gaussians dataset, however pima-indians and wine datasets are also highly variable with a standard deviation of $58.37$ and $215.75$ respectively. The~datasets were normalized using standardization defined in equation \ref{eq:stand}, where $\mu$ is a mean vector for of each of the attributes and $\sigma$ is a vector of their standard deviations.

\begin{equation} \label{eq:stand}
\hat{X} = \frac{X-\mu}{\sigma}
\end{equation}

\section{Experiment: Classification} \label{chap:exp:classification}

In this experiment we wanted to see how much does learnt metric help in classification task compared to standard Euclidean distance. We chose \ac{knn} classifier, which is one of the simplest and yet powerful classifiers and it is very easy to modify to use a custom metric as shown in @REF.
%TODO how do we modify kNN to use custom metric? [here or earlier???]

%TODO explain that "Covariance" refers to Mahalanobis distance with Cov matrix (probably in related work)
The metric learning algorithms that we tested in our experiments are Covariance, \ac{lmnn}, \ac{nca} and \ac{jdefme}. All these methods were described in chapter \ref{chap:rw}. Next, we tested \ac{cmaesknn} that we proposed in chapter \ref{chap:our-method}. We were also curious if the \ac{knn} fitness function would improve \ac{jde} strategy and how would \ac{cmaes} perform with \ac{fme} fitness function so we also tested those. Finally, we also tested all datasets against the Euclidean distance which acted as our baseline method.

To assess performance of the metric learning algorithms we used datasets listed in section \ref{chap:exp:datasets} and used 10-fold cross validation for each algorithm. For each fold data were first preprocessed and standardized as described in section \ref{chap:exp:preprocessing}, then the metric was trained using one of the methods and finally the metric was evaluated using \ac{knn} algorithm.

We also wanted to test, how does standardization influence the learnt metrics and whether the metric learning algorithms can handle unnormalized data and so we also tried all of the experiments again, but without standardizing the data.

\subsection{Hyperperameter search} \label{chap:exp:hypsearch}

Every metric learning algorithm has a set of hyperparameters which highly influences the learnt metric. For each hyperparameter we selected sensible values from a reasonable range. All hyperparameters for each algorithm are shown in table \ref{tab:hyperparams}. The performance of \ac{knn} classifier hugely depends on its number of neighbours hyperparameter. We chose to try values 1, 2, 4, 8, 16, 32, 64, 128 for this parameter. From the hyperparameter table we can see that all the algorithms except Covariance and \ac{lfda} are iterative (max\_iter and n\_gen hyperparameters) and we tried 50, 100, 250 and 1000 iterations for all these iterative algorithms. The transformer parameter means whether the algorithm uses the full matrix or it is restricted to a diagonal transformation matrix. Unfortunately the library did not provide means for trying this and so we tested this only with the evolution methods that we implemented ourselves. The rest of the hyperparameters is self-explanatory and has been described in chapters \ref{chap:rw} and \ref{chap:our-method}.

\begin{table}[ht] \centering
\begin{tabular}{lrl}
\hline
Method & Parameter name & Parameter values \\
\hline

\ac{knn} classifier
    & n\_neighbors & 1, 2, 4, 8, 16, 32, 64, 128  \\

\ac{lmnn}
    & k & 1, 2, 4, 8, 16, 32  \\
    & regularization & 0.1, 0.5, 0.9  \\
    & max\_iter & 50, 250, 500, 1000  \\
    & learn\_rate & 1e-7, 1e-8, 1e-9  \\

\ac{nca}
    & max\_iter & 50, 250, 500, 1000  \\
    & learn\_rate & 0.1, 0.01  \\

\ac{lfda}
    & metric & weighted, orthonormalized  \\

\ac{cmaesknn}
    & transformer & full, diagonal  \\
    & n\_gen & 50, 100, 250, 1000  \\
    & knn\_neighbors & 1, 4, 8, 16  \\
    & knn\_weights & uniform, distance  \\

\ac{jdeknn}
    & transformer & full, diagonal  \\
    & n\_gen & 50, 100, 250, 1000  \\
    & knn\_neighbors & 1, 4, 8, 16  \\
    & knn\_weights & uniform, distance  \\

\ac{cmaesfme}
    & transformer & full, diagonal  \\
    & n\_gen & 50, 100, 250, 1000  \\

\ac{jdefme}
    & transformer & full, diagonal  \\
    & n\_gen & 50, 100, 250, 1000  \\

% ITML
%     & num\_constraints & 10, 100, 1000, 10000  \\
%     & gamma & 0.01, 0.1, 1.0, 10.  \\
%     & max\_iters & 50, 250, 500, 1000  \\
% SDML
%     & num\_constraints & 10000, 100000  \\
%     & use\_cov & True, False  \\
%     & balance\_param & 0.1, 0.25, 0.5, 0.75, 1  \\
%     & sparsity\_param & 0.01, 0.05, 0.1, 0.25  \\
% LSML
%     & num\_constraints & 100, 1000, 10000, 100000  \\
%     & max\_iter & 50, 250, 500, 1000  \\
% RCA
%     & num\_chunks & 10, 50, 100, 500, 1000  \\
%     & chunk\_size & 1, 2, 3, 5, 7, 10, 16, 32  \\

\hline
\end{tabular}
\caption{Values of hyperparameters used for each algorithm} \label{tab:hyperparams}
\end{table}


Even for these sensibly chosen hyperparameter values there are thousands of possible combinations for every single algorithm and thus we have to fit thousands of different models. Let's consider \ac{lmnn} algorithm: there are 10 folds, 8 possible values for the final \ac{knn} classifier hyperparameter, \ac{lmnn} itself has 4 distinct hyperparameters with 6, 3, 4 and 3 values, which means there would be $10*8*(6*3*4*3)=17280$ different models to train just for \ac{lmnn} algorithm if a simple grid search was used. This is a major problem because some of the algorithms take a long time to train even on small datasets and trying that many combinations would take too long to test.

We can notice, however, that we do not need to retrain our metric for every hyperparameter of the final \ac{knn} classifier. Instead we can train a metric model first and then train and test \ac{knn} classifier with all hyperparameter options. Therefore, we designed our version of grid search that is hierarchical in a sense that it will train a metric with all combinations of hyperparameters first and only then it will evaluate these metrics using \ac{knn} with its own hyperparameters. There will still be the same total number of models, however there will be 8 times less evaluations of every metric learning algorithm, which roughly translates to 8 times less computation time as training \ac{knn} classifier is negligible compared to a metric model.

%TODO add a sketch of hierarchical grid search

\subsection{Results}

In table \ref{tab:error-rates} you can see the error rates and their standard deviation (10-fold cross validation was performed) of the final \ac{knn} classifier for every dataset and every method. If the method name is prefixed with "s:" it means the dataset was standardized as described in section \ref{chap:exp:preprocessing}. The bold numbers signify that it is the best result for the given dataset. \ac{nca} did not finish on the 3 largest datasets and so the corresponding cells are marked as a "Timeout". Some other methods did not finish because an error occurred and those cells are marked with an "Error" text. The Covariance method failed for some datasets because the matrix was not \ac{pd} and so it was not possible to use Cholesky decomposition. \ac{lfda} also failed for some datasets because of a factorization of a matrix during the calculation. \ac{nca} failed for gaussians and mice-protein datasets because the numbers overflew to infinity during the computation.

% How hyperparameters influence the successrate. [graphs] %TODO

\begin{table}[ht] \centering
{\small\renewcommand{\arraystretch}{1.0}
\setlength{\tabcolsep}{2pt}
% \hspace*{-36.73627pt}
% \multicolumn{4}{c}{Item} \\
% \cline{1-2}
\begin{tabular}{rcccccccccc}
\toprule


& \multicolumn{1}{c}{balance-scale} & \multicolumn{1}{c}{breast-cancer} & \multicolumn{1}{c}{digits6} & \multicolumn{1}{c}{digits10} & \multicolumn{1}{c}{digits10} \\ 
\midrule
Euclidean & $10.56\pm1.07$ & $3.58\pm2.15$ & $\bm{0.09\pm0.27}$ & $1.22\pm0.86$ & $1.22\pm0.86$ \\
s:Euclidean & $9.92\pm1.18$ & $3.43\pm1.72$ & $0.55\pm0.74$ & $2.06\pm0.90$ & $2.06\pm0.90$ \\
Covariance & $10.09\pm1.55$ & $5.29\pm1.79$ & E  & E  & E  \\
s:Covariance & $10.09\pm1.38$ & $5.29\pm1.79$ & E  & E  & E  \\
LMNN & $5.78\pm2.25$ & $2.72\pm1.63$ & $\bm{0.09\pm0.27}$ & $\bm{0.94\pm1.03}$ & $\bm{0.94\pm1.03}$ \\
s:LMNN & $5.45\pm2.42$ & $3.00\pm2.07$ & $0.28\pm0.42$ & $1.00\pm0.65$ & $1.00\pm0.65$ \\
NCA & $4.02\pm2.12$ & $3.29\pm2.03$ &  &  &  \\
s:NCA & $4.03\pm2.26$ & $3.15\pm1.66$ &  &  &  \\
LFDA & $7.38\pm2.25$ & $3.01\pm2.08$ & E  & E  & E  \\
s:LFDA & $7.06\pm2.14$ & $\bm{2.43\pm1.57}$ & E  & E  & E  \\
CMAES.kNN & $3.55\pm2.20$ & $2.58\pm1.67$ & $0.18\pm0.36$ & $1.00\pm0.89$ & $1.00\pm0.89$ \\
s:CMAES.kNN & $\bm{3.38\pm2.37}$ & $2.58\pm1.90$ & $0.64\pm0.72$ & $1.61\pm0.91$ & $1.61\pm0.91$ \\
CMAES.fMe & $9.61\pm2.31$ & $3.00\pm1.49$ &  & $1.56\pm0.82$ & $1.56\pm0.82$ \\
s:CMAES.fMe & $9.75\pm3.11$ & $3.15\pm2.28$ &  &  &  \\
jDE.fMe & $8.80\pm2.78$ & $3.15\pm1.55$ &  &  &  \\
s:jDE.fMe & $8.66\pm3.03$ & $3.01\pm1.76$ &  &  &  \\
jDE.kNN & $3.55\pm2.20$ & $2.72\pm1.86$ &  &  &  \\
s:jDE.kNN & $3.55\pm2.20$ & $2.86\pm1.57$ &  &  &  \\
\midrule
& \multicolumn{1}{c}{iris} & \multicolumn{1}{c}{mice-protein} & \multicolumn{1}{c}{pima-indians} & \multicolumn{1}{c}{sonar} & \multicolumn{1}{c}{wine} \\ 
\midrule
Euclidean & $2.67\pm3.27$ & $0.47\pm0.64$ & $24.61\pm3.84$ & $17.88\pm11.43$ & $22.80\pm9.19$ \\
s:Euclidean & $3.33\pm3.33$ & $0.38\pm0.63$ & $24.74\pm3.91$ & $13.50\pm10.10$ & $1.64\pm2.50$ \\
Covariance & $8.00\pm6.53$ & E  & $26.43\pm3.84$ & $17.70\pm6.90$ & $6.21\pm6.78$ \\
s:Covariance & $8.00\pm6.53$ & E  & $26.43\pm3.84$ & $17.70\pm6.90$ & $6.21\pm6.78$ \\
LMNN & $\bm{1.33\pm2.67}$ & $\bm{0.00\pm0.00}$ & $24.09\pm5.15$ & $13.53\pm6.18$ & $3.83\pm4.24$ \\
s:LMNN & $2.00\pm3.06$ & $\bm{0.00\pm0.00}$ & $23.70\pm4.11$ & $\bm{9.19\pm6.67}$ & $\bm{0.56\pm1.67}$ \\
NCA & $2.00\pm3.06$ &  & E  & $14.41\pm8.86$ & E  \\
s:NCA & $2.00\pm3.06$ &  & $22.53\pm4.19$ & $12.00\pm5.42$ & $2.29\pm3.75$ \\
LFDA & $2.00\pm3.06$ & E  & $23.70\pm3.28$ & $14.37\pm7.02$ & $1.64\pm2.50$ \\
s:LFDA & $2.00\pm3.06$ & E  & $23.06\pm4.71$ & $13.50\pm10.10$ & $1.11\pm2.22$ \\
CMAES.kNN & $\bm{1.33\pm2.67}$ & $\bm{0.00\pm0.00}$ & $23.05\pm4.03$ & $13.48\pm8.57$ & $3.37\pm4.55$ \\
s:CMAES.kNN & $2.00\pm3.06$ & $0.10\pm0.29$ & $\bm{21.48\pm3.13}$ & $10.17\pm6.39$ & $2.25\pm3.72$ \\
CMAES.fMe & $\bm{1.33\pm2.67}$ &  & $23.83\pm4.11$ &  & $12.90\pm8.52$ \\
s:CMAES.fMe & $2.00\pm4.27$ &  & $21.61\pm4.03$ &  &  \\
jDE.fMe & $2.00\pm3.06$ &  & $23.30\pm3.58$ &  &  \\
s:jDE.fMe & $3.33\pm4.47$ &  & $22.40\pm4.69$ &  &  \\
jDE.kNN & $2.00\pm3.06$ &  & $23.83\pm3.76$ &  &  \\
s:jDE.kNN & $2.67\pm3.27$ &  & $23.18\pm3.89$ &  &  \\


\bottomrule
\end{tabular}
}
\caption{Error rates (in \%) for every method and dataset} \label{tab:errors-small}
\end{table}


% [Comparison to unstandardized data]
% can metric learning replace standardization? %TODO
% comparison discussion... %TODO

% https://en.wikipedia.org/wiki/Interquartile_range
The results are also plotted in figures \ref{fig:cl:sr1} and \ref{fig:cl:sr2} using the box plots. Each dataset has its own graph where all the methods that finished correctly are drawn with their specific colour using the box plot graph. The median success rate is marked with a thick black line, the area of the box is the \ac{iqr} defined as the third quartile minus first quartile: $IQR = Q_3 - Q_1$. The whiskers extend $1.5*IQR$ to both sides of the box plot and anything beyond whiskers is considered to be an outlier value.

%TODO reults discussion

\cenfig{graphs/classification/sr_1.pdf}{Boxplots of successrates from 10-fold crossvalidation for the first 6 datasets}{fig:cl:sr1}

\cenfig{graphs/classification/sr_2.pdf}{Boxplots of successrates from 10-fold crossvalidation for the last 4 datasets}{fig:cl:sr2}

To see how the hyperparameters influence the success rate of the \ac{knn} classifier we choose one hyperparameter and for each of its values we look at the best possible success rate among the rest of the hyperparameters.

First we look at the $k$ hyperparameter in the final \ac{knn} classifier. In figure \ref{fig:cl:kpar} there is one graph for every dataset (only digits6 dataset is missing because it behaves very similarly to digits10 dataset) in which one line corresponds to one algorithm. From individuals graphs it is possible to see some trends among all the methods, in particular that the lines tend to copy a downwards parabola shape, meaning that $k$ values of 1 and 2 are too small and do not generalize well and values 64 and 128 are too big for these small datasets. The best shape for any particular method would be a flat line near the top of the graph, which would imply that the method separated the instances into a well separable and homogeneous clusters in which all instances have the same label. Unfortunately none of the methods behave so great that they would have flat graphs. Nevertheless, what is important to notice, is that some of the methods seem to handle the change of this parameters more gracefully. This is the most prominent with \ac{lmnn} and \ac{nca} methods in the mice-protein dataset: these two methods have large success rates even when the $k$ parameter increases, whereas the other method success rates drop significantly. This fenomenon is visible in other datasets and even other methods, for example in gaussians dataset for the \ac{cmaesknn} as well as the two already mentioned. The Covariance method has a bad success rate altogether, but it also drops down the fastest when the $k$ hyperparameter increases.

%TODO refresh graph, change datasets!
\cenfig{graphs/classification/sr_knn}{Successrates of individual algorithms when fixing `k` in the final kNN classifier}{fig:cl:kpar}

\cenfig{graphs/classification/sr_hyp}{Successrates of individual algorithms when fixing some of their hyperparameters}{fig:cl:hyp}

\section{Experiment: Generalization of EAs} \label{chap:exp:fitness}

To test out how well do solutions from evolution algorithms evolve and generalize we performed another series of tests. Using CMAES and jDE evolution strategies described in sections \ref{chap:ea:cmaes} and \ref{chap:ea:jde}, combined with k-Nearest~Neighbour (kNN) and weighted~F-measure (wFme) fitnesses described in sections @WHERE and \ref{chap:rw:fukui} respectively. Both strategies with both fitnesses were tested, giving us 4 different models: CMAES.kNN, CMAES.wFme, jDE.kNN and jDE.wFme where the first part until the dot describes evolution strategy used in the algorithm and the part after the dot describes its fitness function.

All methods have some hyperparameters and in order to avoid the time consuming hyperparameter search, the hyperparameters for the models in this experiment were picked according to results from the previous experiment. Only one set of hyperparameters was picked for all datasets. The number of nearest neighbours in the kNN classifier used in the fitness function was chosen to be 8 and the number of nearest neighbours in the final kNN classifier was set to 4. All methods were evolved for a full thousand generations. Early stopping does not make sense in this experiment because the learnt metric is evaluated after every single generation, compared to the classification experiment where more generations could overfit the testing error and therefore the produced metric would not generalize well.

[full vs diagonal!?]
[some datasets too large]

Datasets described in section \ref{chap:exp:datasets} were used as well for this experiment. Data were preprocessed and standardized the same way as previous experiment, both methods are described in section \ref{chap:exp:preprocessing}. Data were split into training and testing sets using a 67:33 ratio in a stratified fashion, keeping the sizes of classes balanced between the two splits.

To gauge the performance of the fitness function we logged the fitness values of all individuals in every generation and then calculated median, minimum, maximum, 25th and 75th percentiles in each generation.

To see how well the individuals generalize, individuals with the best and the worst fitness values were picked in every generation and the metric encoded in these individuals was evaluated using the test set using kNN classifier.

\subsection{Results}

Both kNN and wFme fitness functions output values in a closed range $[0,1]$. The final kNN classifier used to gauge performance outputs values in the same exact range and so we plot all values in the same graph. Therefore the graph for each generation shows minimal and maximal (both in orange) fitnesses, median fitness (in green) with the area between 25th and 75th percentiles coloured with light green. The test successrate of the individual with the highest fitness in XX and the test successrate of the individual with the lowest fitness in XX.

[each dataset has 4 graphs] %TODO

[explain the graphs] %TODO

[discuss results] %TODO

\cenfig{graphs/fitness/balance-scale}{Fitness evolution and generalization on `balance-scale` dataset}{fig:fitness:balance-scale}
\cenfig{graphs/fitness/breast-cancer}{Fitness evolution and generalization on `breast-cancer` dataset}{fig:fitness:breast-cancer}
\cenfig{graphs/fitness/digits10}{Fitness evolution and generalization on `digits10` dataset}{fig:fitness:digits10}
\cenfig{graphs/fitness/digits6}{Fitness evolution and generalization on `digits6` dataset}{fig:fitness:digits6}
\cenfig{graphs/fitness/gaussians}{Fitness evolution and generalization on `gaussians` dataset}{fig:fitness:gaussians}
\cenfig{graphs/fitness/iris}{Fitness evolution and generalization on `iris` dataset}{fig:fitness:iris}
\cenfig{graphs/fitness/mice-protein}{Fitness evolution and generalization on `mice-protein` dataset}{fig:fitness:mice-protein}
\cenfig{graphs/fitness/pima-indians}{Fitness evolution and generalization on `pima-indians` dataset}{fig:fitness:pima-indians}
\cenfig{graphs/fitness/sonar}{Fitness evolution and generalization on `sonar` dataset}{fig:fitness:sonar}
\cenfig{graphs/fitness/wine}{Fitness evolution and generalization on `wine` dataset}{fig:fitness:wine}


\section{Experiment: Learning time} \label{chap:exp:learning-times}

Learning time of each method is an important factor to be considered. Some methods are not scalable for large datasets as already discussed in chapter \ref{chap:rw}. It is not easy to describe the metric learning algorithms in the big O notation because most of the methods are iterative and it is unclear how many iterations is needed for them to converge. Moreover, some of them can still end up in a local minima thus multiple runs may be necessary for optimal results.

In this experiment we measured the learning times of all methods (the same methods as in previous experiments: Covariance, \ac{lmnn}, \ac{nca}, \ac{lfda} and both evolution strategies with both evolution fitnesses: \ac{cmaesknn}, \ac{cmaesfme}, \ac{jdeknn} and \ac{jdefme}. For the evolution methods we also measured learning only a restricted diagonal transformation matrix. All methods were run 10 times on every dataset with a set of the hyper-parameters which corresponds to best error rate for given dataset from the first experiment in section \ref{chap:exp:classification}. Euclidean distance does not need any training and therefore is omitted from this experiment altogether. The experiment was performed on a single computer so that the results would be comparable. Having this restriction, the maximal time for the method to finish was set to one hour, after this period the experiment was interrupted. The methods that did not finish are marked as "Timeout".

The results are in table \ref{tab:learning-times} where the learning times shown are averaged across all 10 runs. Calculating the Covariance metric is the fastest, closely followed by \ac{lfda} algorithm. Some of the datasets were so small that these two methods would take a less than a hundredth of a second. On the other hand, \ac{lmnn} and \ac{nca} metrics and all the evolution metrics take a lot of time to learn. Generally, \ac{jde} evolution strategy should be much faster compared to \ac{cmaes} strategy because calculating a covariance matrix inside \ac{cmaes} is expensive, however the \ac{jde} needs a much bigger population and the fitness evaluation is very expensive in our case as discussed in @REF. Therefore, \ac{jde} is clearly much slower compared to \ac{cmaes} strategy in the results table. It is also clear that the \ac{fme} fitness is much slower compared to \ac{knn} fitness as discussed in @REF when we compare \ac{cmaesknn} with \ac{cmaesfme} and \ac{jdeknn} with \ac{jdefme}. This makes the method \ac{jdefme} described in \cite{fukui2013evolutionary} the slowest method in all of our benchmarked methods and is even several times slower than \ac{nca} algorithm. The \ac{cmaesknn} performs comparable to \ac{lmnn} in terms of learning time. The restricted versions of the evolutionary algorithms to a diagonal transformation matrix learn much faster thanks to smaller individuals which directly encode the Mahalanobis matrix and also the smaller populations as also discussed in @REF. What is important to note, however, is that differences between learning a full and a diagonal matrices using \ac{cmaes} are much smaller than the differences between learning a full and a diagonal matrices using \ac{jde} algorithm, where for \ac{cmaes} the difference is about 2 times, but for \ac{jde} the difference is much more significant, even 10 times slower on some datasets.

%TODO fix missing values
\begin{table}[ht] \centering
{\small\renewcommand{\arraystretch}{1.0}
\setlength{\tabcolsep}{2pt}
\begin{tabular}{rcccccccccc}
\toprule


& \multicolumn{1}{c}{balance-scale} & \multicolumn{1}{c}{breast-cancer} & \multicolumn{1}{c}{digits6} & \multicolumn{1}{c}{digits10} & \multicolumn{1}{c}{gaussians} \\ 
\midrule
s:Covariance & $\bm{0.01}$ & $\bm{0.00}$ & $\bm{0.00}$ & $\bm{0.01}$ & $\bm{0.00}$ \\
s:LMNN & $2.78$ & $10.78$ & Timeout  & $858.98$ & $1.77$ \\
s:NCA & $71.37$ & $176.38$ & Timeout  & Timeout  & $38.04$ \\
s:LFDA & $0.02$ & $0.03$ & $0.02$ & $0.03$ & $0.01$ \\
s:CMAES.kNN (full) & $5.76$ & $9.20$ & Timeout  & Timeout  & $5.02$ \\
s:CMAES.fMe (full) & $17.73$ & $15.80$ & Timeout  & Timeout  & $15.43$ \\
s:jDE.fMe (full) & $206.34$ & $648.13$ & Timeout  & Timeout  & $166.43$ \\
s:jDE.kNN (full) & $77.31$ & $495.96$ & Timeout  & Timeout  & $56.14$ \\
s:CMAES.kNN (diag) & $3.81$ & $5.35$ & $26.86$ & $61.63$ & $3.12$ \\
s:CMAES.fMe (diag) & $8.04$ & $10.01$ & $42.44$ & $116.62$ & $9.73$ \\
s:jDE.fMe (diag) & $30.64$ & $69.18$ & $1577.14$ & $2951.39$ & $35.63$ \\
s:jDE.kNN (diag) & $16.48$ & $52.55$ & $2230.37$ & $4578.65$ & $13.47$ \\
\midrule
& \multicolumn{1}{c}{iris} & \multicolumn{1}{c}{mice-protein} & \multicolumn{1}{c}{pima-indians} & \multicolumn{1}{c}{sonar} & \multicolumn{1}{c}{wine} \\ 
\midrule
s:Covariance & $\bm{0.00}$ & $\bm{0.00}$ & $\bm{0.00}$ & $\bm{0.00}$ & $\bm{0.00}$ \\
s:LMNN & $0.09$ & Timeout  & $12.99$ & $7.87$ & $2.34$ \\
s:NCA & $10.47$ & Timeout  & $164.51$ & $880.07$ & $21.90$ \\
s:LFDA & $\bm{0.00}$ & $0.20$ & $0.02$ & $0.04$ & $0.01$ \\
s:CMAES.kNN (full) & $4.67$ & Timeout  & $7.56$ & $3165.69$ & $8.46$ \\
s:CMAES.fMe (full) & $9.83$ & Timeout  & $18.78$ & $3481.85$ & $18.54$ \\
s:jDE.fMe (full) & $104.10$ & Timeout  & $644.15$ & Timeout  & $1404.01$ \\
s:jDE.kNN (full) & $46.97$ & Timeout  & $380.90$ & Timeout  & $696.73$ \\
s:CMAES.kNN (diag) & $2.80$ & $37.04$ & $4.54$ & $6.48$ & $4.02$ \\
s:CMAES.fMe (diag) & $6.59$ & $65.79$ & $12.94$ & $17.05$ & $10.29$ \\
s:jDE.fMe (diag) & $23.15$ & $2619.58$ & $78.49$ & $549.09$ & $92.20$ \\
s:jDE.kNN (diag) & $11.27$ & $3561.06$ & $41.83$ & $285.22$ & $40.19$ \\


\bottomrule
\end{tabular}
}
\caption{Learning times (in seconds) for each method on every dataset} \label{tab:learning-times}
\end{table}


To further investigate how well do metric learning methods scale, we also designed two additional experiments to investigate the learning times. In the first experiment we kept the dimensionality of the data fixed to 5 and measured the learning times for the number of samples in the range from 100 to 1500 with 100 increments. In the second experiment we fixed the number of samples to 500 and measured the learning times for all methods with the increasing dimension in the range from 2 to 10 with an increment of 1. All these measurements were done using the digits10 dataset from which the required number of samples was sampled using the stratified sampling and then the dimension of the samples was reduced using \ac{pca} to a required dimension.

Both experiments are represented in figures \ref{fig:learning-times:samples} and \ref{fig:learning-times:dimensions}. Most of the methods merge together in the graph so we show two graphs for each experiment: one with linear and one with logarithmic scale applied on y axis. Covariance and \ac{lfda} algorithms are not included in these two figures because they perform so fast on this experiment that the results are not interpretable due to random fluctuations in the measurements.

From figure \ref{fig:learning-times:samples} it is clear that the \ac{jde} algorithm performs the worst and scales poorly with an increasing number of samples with both \ac{knn} and \ac{fme} fitnesses. \ac{nca} performs also very bad with an increasing number of samples. On the other hand the rest of the methods perform relatively well and are indistinguishable on the linearly scaled axis. \ac{lmnn} and \ac{cmaes} algorithms have a similar performance, however \ac{lmnn} also performs much slower with more and more instances. In contrast, \ac{cmaes} algorithm scales very well and the number of samples hardly influences the performance. Looking at the evolutionary methods restricted to a diagonal matrix, it is clear that \ac{cmaesknn} (diag) is the fastest among all of them, however it is not much slower compared to its unrestricted version \ac{cmaesknn} (full). Restricted version of \ac{jde} algorithms are much faster compared to their unrestricted counterparts, but even these restricted versions are slower and scale worse than \ac{cmaesknn} (full).

\cenfig{graphs/learning-times/samples}{Learning times for increasing number of samples with a fixed dimension (normal and log scale)}{fig:learning-times:samples}

Similarly, from figure \ref{fig:learning-times:dimensions} we can see that \ac{jde} scales very poorly with an increasing dimensionality of the dataset as well. The order of the methods in the graphs remained the same as in the previous experiment, however there are some notable changes in the shapes of the curves. \ac{lfda} algorithm scales very well with an increasing dimensionality; \ac{nca} and \ac{lmnn} are not as steep as in the previous experiment, but they still scale worse compared to \ac{cmaes} algorithm. It is clear that restricted versions of all evolutionary algorithms scale much better in regards of the increasing dimensionality because the dimensionality directly reflects in the individuals size: for the restricted version the individuals have size $D$, but for the unrestricted version we need to encode the whole matrix of size $D^2$. The dimensionality of the individuals directly reflect in the population size as discussed in @REF. Therefore the restricted versions scale better with the increasing dimension and their graphs are less steep. When the steepness of \ac{cmaes} and \ac{jde} algorithms is directly compared however, the \ac{jde} scales worse even in the restricted version. That is due to the population size that the algorithm uses by default being linear, compared to \ac{cmaes}, which only has a population of a logarithmic size of its individual dimensionality as already discussed in \ref{chap:es:cmaes}.

\cenfig{graphs/learning-times/dimensions}{Learning times for increasing dimension with a fixed number of samples (normal and log scale)}{fig:learning-times:dimensions}


\section{Experiment: Dimensionality reduction} \label{chap:exp:dimred}

Metric learning can also be used for dimensionality reduction as discussed in @WHERE. This experiment was focused on dimensionality reduction into two dimensions where it is easy to visualise the data.

One of the most common and the simplest unsupervised dimensionality reduction method is \ac{pca} @REFERENCE. \ac{pca} picks the dimensions of the data with the highest variance. This corresponds to finding the eigenvalues and eigenvectors of the covariance matrix, the eigenvectors with the largest eigenvalues correspond to the dimensions that have the strongest correlation in the dataset.

[Is Covariance Mahalanobis === PCA??]

LFDA uses the same trick for dimensionality reduction, it calculates the metric matrix and then picks the columns of this matrix with the largest corresponding eigenvalues.

\cenfig{graphs/dimred/gaussians}{Dimensionality reduction of multivariate Gaussian dataset using different algorithms}{fig:dimred:gauss}

See fig \ref{fig:dimred:gauss}

\subsection{Direct comparison with other methods}

In this experiment we compare PCA, LFDA, CMAES.kNN methods for dimensionality reduction into two dimensions.


\subsection{Neural Network transformation}

...

\subsection{Preprocessing for t-SNE}

t-SNE \cite{maaten2008visualizing} is another unsupervised dimensionality reduction method which is getting a lot of attention recently. Compared to PCA, t-SNE has a non-convex objective function and is optimized using gradient descend, which means it may end up in a local minimum. .... recommended to run more times.

[add how tsne works?]

In this experiment we compare pure t-SNE with

%%%%%%%%%%%%%%%%%%%%%%%%%%%%%%%%%%%%%%%%%%%%%%%%%%%%%%%%%%%%%%%%%%%%%

\chapter{Implementation} \label{chap:impl}

\section{Hierarchical Grid search}

how it works - scheme

how cross-validation is in memory

\section{Metric learning in metric\_learn package}

Description of modules, fitnesses, transformers, ...

\begin{figure}[h!] \label{fig:implementation-modules}
	\centering
    \includegraphics[width=0.2\textwidth]{img/notfound}
    \caption{Interaction between modules for Metric evolution}
\end{figure}

%%%%%%%%%%%%%%%%%%%%%%%%%%%%%%%%%%%%%%%%%%%%%%%%%%%%%%%%%%%%%%%%%%%%%

\chapter{Discussion}

Graphs, performace, caveats, what worked, what didnt work...

LMNN, ITML, SDML take too long to calculate

LMNN and LFDA perform very well

LMNN has too many hyper parameters

Simple Covariance metric was impossible to calculate on certain datasets

%%%%%%%%%%%%%%%%%%%%%%%%%%%%%%%%%%%%%%%%%%%%%%%%%%%%%%%%%%%%%%%%%%%%%

\chapter*{Conclusion}
\addcontentsline{toc}{chapter}{Conclusion}

This worked, this didn't work. What we did.

\chapter*{Future work}

What wasn`t finished and should be.

%%%%%%%%%%%%%%%%%%%%%%%%%%%%%%%%%%%%%%%%%%%%%%%%%%%%%%%%%%%%%%%%%%%%%
%%%%%%%%%%%%%%%%%%%%%%%%%%%%%%%%%%%%%%%%%%%%%%%%%%%%%%%%%%%%%%%%%%%%%
%%%%%%%%%%%%%%%%%%%%%%%%%%%%%%%%%%%%%%%%%%%%%%%%%%%%%%%%%%%%%%%%%%%%%

%%% Bibliography
\include{bibliography}

%%% Figures used in the thesis (consider if this is needed)
\listoffigures

%%% Tables used in the thesis (consider if this is needed)
%%% In mathematical theses, it could be better to move the list of tables to the beginning of the thesis.
\listoftables

%%% Abbreviations used in the thesis, if any, including their explanation
%%% In mathematical theses, it could be better to move the list of abbreviations to the beginning of the thesis.
\chapwithtoc{List of Abbreviations}

\printacronyms[include-classes=abbrev,heading=none] % ,name=Abbreviations

\chapwithtoc{List of Notations}
% More inspiration:
% http://www.deeplearningbook.org/contents/notation.html

% \begin{table}[ht] \centering
% \caption{Summary of the main notations} \label{tab:notation}
% \begin{tabular}{ll}
% \hline
% Notation & Description \\
% \hline

\begin{tabbing}
\hspace{150pt}\=\kill

$\mathbb{R}$ \> Set of real numbers \\
$\mathbb{R}^d$ \> Set of d-dimensional real-valued vectors \\
$\mathbb{R}^{c \times d}$ \> Set of c$\times$d real-valued matrices \\
$\mathbb{S}^{d}_+$ \> Cone of symmetric PSD $d \times d$ real-valued matrices \\

$\textbf{x}$ \> An arbitrary vector \\
$\bm{M}$ \> An arbitrary matrix \\
$\textbf{\textit{I}}$ \> Identity matrix \\
$\bm{M} \succ 0$ \> PD matrix M \\
$\bm{M} \succeq 0$ \> PSD matrix M \\
$\parallel \cdot \parallel_p$ \> p-norm \\
$\parallel \cdot \parallel_\mathcal{F}$ \> Frobenius norm \\
$\parallel \cdot \parallel_*$ \> Nuclear norm \\
$\tr(\bm{M})$ \> Trace of matrix M \\
$[t]_+ = max(0, 1-t)$ \> Hinge loss function \\
$\xi$ \> Slack variable \\
$\Sigma$ \> Finite alphabet \\

$\mathcal{X} \subseteq \mathbb{R}^d$ \> Input (instance) space \\
$\mathcal{Y} = \{ 1, \ldots ,c \}$ \> Set of $c$ output labels \\
$z=(x,y) \in \mathcal{X} \times \mathcal{Y}$ \> An arbitrary labeled instance \\

$\mathcal{S} = \{ (x_i, x_j) \mid y_i = y_j \}$ \> Set of must-link constraints \\
$\mathcal{D} = \{ (x_i, x_j) \mid y_i \neq y_j \}$ \> Set of cannot-link constraints \\

$d_{\bm{2}}$ \> Euclidean distance \\
$d_{\bm{M}}$ \> Mahalanobis distance parametrized by $\bm{M}$ \\
% $\mathcal{R} = \{ (x_i, x_j, x_k) \mid y_i = y_j \wedge y_i \neq y_k \}$ \> Set of relative constraints \\


% $\widehat{\mathcal{Y}} = \{ 1, \ldots ,\widehat{c} \} \subseteq \mathcal{Y}$ \> Set of $\widehat{c}$ known output labels \\
% $x$ \> String of finite size
% $\mathcal{L}=\{z_i=(x_i, y_i)\}^n_{i=1}$ \> Set of $n$ labeled training instances \\
% $\mathcal{U} = \{x_i \}^m_{i=1}$ \> Set of $m$ unlabeled training instances \\
% $\overline{z} = (x, \overline{y}) \in \mathcal{X} \times \widehat{\mathcal{Y}} $ \> An arbitrary instance with predicted label \\
% $\overline{\mathcal{Z}} = \{\overline{z_i} = (x_i, \overline{y_i}) \}^m_{i=1}$ \> Set of $m$ training instances with predicted labels \\

% \hline
% \end{tabular}
% \end{table}

\end{tabbing} 


%%% Attachments to the master thesis, if any. Each attachment must be
%%% referred to at least once from the text of the thesis. Attachments
%%% are numbered.
%%%
%%% The printed version should preferably contain attachments, which can be
%%% read (additional tables and charts, supplementary text, examples of
%%% program output, etc.). The electronic version is more suited for attachments
%%% which will likely be used in an electronic form rather than read (program
%%% source code, data files, interactive charts, etc.). Electronic attachments
%%% should be uploaded to SIS and optionally also included in the thesis on a~CD/DVD.
\chapwithtoc{Attachments}

\openright
\end{document}
