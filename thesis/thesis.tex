%%% The main file. It contains definitions of basic parameters and includes all other parts.

%% Settings for single-side (simplex) printing
% Margins: left 40mm, right 25mm, top and bottom 25mm
% (but beware, LaTeX adds 1in implicitly)
\documentclass[12pt,a4paper]{report}
\setlength\textwidth{145mm}
\setlength\textheight{247mm}
\setlength\oddsidemargin{15mm}
\setlength\evensidemargin{15mm}
\setlength\topmargin{0mm}
\setlength\headsep{0mm}
\setlength\headheight{0mm}
% \openright makes the following text appear on a right-hand page
\let\openright=\clearpage

%% Settings for two-sided (duplex) printing
% \documentclass[12pt,a4paper,twoside,openright]{report}
% \setlength\textwidth{145mm}
% \setlength\textheight{247mm}
% \setlength\oddsidemargin{14.2mm}
% \setlength\evensidemargin{0mm}
% \setlength\topmargin{0mm}
% \setlength\headsep{0mm}
% \setlength\headheight{0mm}
% \let\openright=\cleardoublepage

%% Character encoding: usually latin2, cp1250 or utf8:
\usepackage[utf8]{inputenc}

%% Further useful packages (included in most LaTeX distributions)
\usepackage{amsmath}        % extensions for typesetting of math
\usepackage{amsfonts}       % math fonts
\usepackage{amsthm}         % theorems, definitions, etc.
\usepackage{bbding}         % various symbols (squares, asterisks, scissors, ...)
\usepackage{bm}             % boldface symbols (\bm)
\usepackage{graphicx}       % embedding of pictures
\usepackage{fancyvrb}       % improved verbatim environment
\usepackage{natbib}         % citation style AUTHOR (YEAR), or AUTHOR [NUMBER]
\usepackage[nottoc]{tocbibind} % makes sure that bibliography and the lists
			    % of figures/tables are included in the table
			    % of contents
\usepackage{dcolumn}        % improved alignment of table columns
\usepackage{booktabs}       % improved horizontal lines in tables
\usepackage{paralist}       % improved enumerate and itemize
\usepackage[usenames]{xcolor}  % typesetting in color

%%% User-defined packages
\usepackage{amssymb}
\usepackage[inline]{enumitem}
\DeclareMathOperator{\tr}{tr}

%%% Basic information on the thesis

% Thesis title in English (exactly as in the formal assignment)
\def\ThesisTitle{Evolutionary Algorithms for Data Transformation}

% Author of the thesis
\def\ThesisAuthor{Ondřej Švec}

% Year when the thesis is submitted
\def\YearSubmitted{2017}

% Name of the department or institute, where the work was officially assigned
% (according to the Organizational Structure of MFF UK in English,
% or a full name of a department outside MFF)
\def\Department{Department of Theoretical Computer Science and Mathematical Logic}

% Is it a department (katedra), or an institute (ústav)?
\def\DeptType{Department}

% Thesis supervisor: name, surname and titles
\def\Supervisor{Mgr. Martin Pilát, Ph.D.}

% Supervisor's department (again according to Organizational structure of MFF)
\def\SupervisorsDepartment{Department of Theoretical Computer Science and Mathematical Logic}

% Study programme and specialization
\def\StudyProgramme{Computer Science}
\def\StudyBranch{Artificial Intelligence}

% An optional dedication: you can thank whomever you wish (your supervisor,
% consultant, a person who lent the software, etc.)
\def\Dedication{%
Dedication. %TODO
}

% Abstract (recommended length around 80-200 words; this is not a copy of your thesis assignment!)
\def\Abstract{%
Abstract. %TODO
}

% 3 to 5 keywords (recommended), each enclosed in curly braces
\def\Keywords{%
{key} {words} %TODO
}

%% The hyperref package for clickable links in PDF and also for storing
%% metadata to PDF (including the table of contents).
\usepackage[pdftex,unicode]{hyperref}   % Must follow all other packages
\hypersetup{breaklinks=true}
\hypersetup{pdftitle={\ThesisTitle}}
\hypersetup{pdfauthor={\ThesisAuthor}}
\hypersetup{pdfkeywords=\Keywords}
\hypersetup{urlcolor=blue}

% Definitions of macros (see description inside)
%%% This file contains definitions of various useful macros and environments %%%
%%% Please add more macros here instead of cluttering other files with them. %%%

%%% Minor tweaks of style

% These macros employ a little dirty trick to convince LaTeX to typeset
% chapter headings sanely, without lots of empty space above them.
% Feel free to ignore.
\makeatletter
\def\@makechapterhead#1{
  {\parindent \z@ \raggedright \normalfont
   \Huge\bfseries \thechapter. #1
   \par\nobreak
   \vskip 20\p@
}}
\def\@makeschapterhead#1{
  {\parindent \z@ \raggedright \normalfont
   \Huge\bfseries #1
   \par\nobreak
   \vskip 20\p@
}}
\makeatother

% This macro defines a chapter, which is not numbered, but is included
% in the table of contents.
\def\chapwithtoc#1{
\chapter*{#1}
\addcontentsline{toc}{chapter}{#1}
}

% Draw black "slugs" whenever a line overflows, so that we can spot it easily.
\overfullrule=1mm

%%% Macros for definitions, theorems, claims, examples, ... (requires amsthm package)

\theoremstyle{plain}
\newtheorem{thm}{Theorem}
\newtheorem{lemma}[thm]{Lemma}
\newtheorem{claim}[thm]{Claim}

\theoremstyle{plain}
\newtheorem{defn}{Definition}

\theoremstyle{remark}
\newtheorem*{cor}{Corollary}
\newtheorem*{rem}{Remark}
\newtheorem*{example}{Example}

%%% An environment for proofs

%%% FIXME %%% \newenvironment{proof}{
%%% FIXME %%%   \par\medskip\noindent
%%% FIXME %%%   \textit{Proof}.
%%% FIXME %%% }{
%%% FIXME %%% \newline
%%% FIXME %%% \rightline{$\square$}  % or \SquareCastShadowBottomRight from bbding package
%%% FIXME %%% }

%%% An environment for typesetting of program code and input/output
%%% of programs. (Requires the fancyvrb package -- fancy verbatim.)

\DefineVerbatimEnvironment{code}{Verbatim}{fontsize=\small, frame=single}

%%% The field of all real and natural numbers
\newcommand{\R}{\mathbb{R}}
\newcommand{\N}{\mathbb{N}}

%%% Useful operators for statistics and probability
\DeclareMathOperator{\pr}{\textsf{P}}
\DeclareMathOperator{\E}{\textsf{E}\,}
\DeclareMathOperator{\var}{\textrm{var}}
\DeclareMathOperator{\sd}{\textrm{sd}}

%%% Transposition of a vector/matrix
\newcommand{\T}[1]{#1^\top}

%%% Various math goodies
\newcommand{\goto}{\rightarrow}
\newcommand{\gotop}{\stackrel{P}{\longrightarrow}}
\newcommand{\maon}[1]{o(n^{#1})}
\newcommand{\abs}[1]{\left|{#1}\right|}
\newcommand{\dint}{\int_0^\tau\!\!\int_0^\tau}
\newcommand{\isqr}[1]{\frac{1}{\sqrt{#1}}}

%%% Various table goodies
\newcommand{\pulrad}[1]{\raisebox{1.5ex}[0pt]{#1}}
\newcommand{\mc}[1]{\multicolumn{1}{c}{#1}}


%%% User-defined commands
% \newcommand{name}[num]{definition}
\newcommand{\cenfig}[3]{
\begin{figure}[ht]
    \centering
    \includegraphics[width=\textwidth,height=\textheight,keepaspectratio]{#1}
    \caption{#2} \label{#3}
\end{figure}
}


\newcommand*\rot{\rotatebox{90}}

\newcolumntype{L}[1]{>{\raggedright\let\newline\\\arraybackslash\hspace{0pt}}m{#1}}
\newcolumntype{C}[1]{>{\centering\let\newline\\\arraybackslash\hspace{0pt}}m{#1}}
\newcolumntype{R}[1]{>{\raggedleft\let\newline\\\arraybackslash\hspace{0pt}}m{#1}}

\DeclareMathOperator*{\argmin}{argmin}
\DeclareMathOperator*{\argmax}{argmax}
\DeclareMathOperator*{\tr}{tr}
\newcommand{\textapproxx}{\raisebox{0.5ex}{\texttildelow}}

\DeclarePairedDelimiter{\ceil}{\lceil}{\rceil}
\DeclarePairedDelimiter{\floor}{\lfloor}{\rfloor}

\newcommand{\twopartdef}[4]
{
	\left\{
		\begin{array}{ll}
			#1 & \mbox{if } #2 \\
			#3 & \mbox{if } #4
		\end{array}
	\right.
}

% define "struts", as suggested by Claudio Beccari in
%    a piece in TeX and TUG News, Vol. 2, 1993.
\newcommand\Tstrut{\rule{0pt}{2.6ex}}         % = `top' strut
\newcommand\Bstrut{\rule[-0.9ex]{0pt}{0pt}}   % = `bottom' strut

\newcommand\numberthis{\addtocounter{equation}{1}\tag{\theequation}}


% Title page and various mandatory informational pages
\begin{document}
\include{title}

%%% A page with automatically generated table of contents of the master thesis

\tableofcontents

%%% Each chapter is kept in a separate file

%%%%%%%%%%%%%%%%%%%%%%%%%%%%%%%%%%%%%%%%%%%%%%%%%%%%%%%%%%%%%%%%%%%%%
%%%%%%%%%%%%%%%%%%%%%%%%%%%%%%%%%%%%%%%%%%%%%%%%%%%%%%%%%%%%%%%%%%%%%

%\chapter*{Introduction}
\addcontentsline{toc}{chapter}{Introduction}

The need for measuring distance or similarity between data is ubiquitous in machine learning and creating custom metrics by hand is very difficult process. This has led to an introduction of distance metric learning where we learn distance metrics or similarities from data themselves.

One of the popular metrics is Mahalanobis distance, which generalizes Euclidean distance with a parameter matrix. We examine methods for learning this Mahalanobis distance and compare their performance on common datasets.

% Úvod do problematiky
% - Co už se udělalo
% - Co je známo
% - Co není známo
% - Co a pro Co a proč chceme naší prací objasnit naší prací objasnit
% - Jasná definice cíle práce
% - Struktura práce

% Related works
% - Co už udělali jiní
% - Co nového hodlá udělat autor

% v úvodu
% samostatná kapitola za úvodem samostatná kapitola za úvodem
% v závěru článku

%\chapter{Methods}

In this work we chose to examine the following methods:

\begin{enumerate}
\item Simple Covariance Metric
\item Xing
\item Large Margin Nearest Neighbor (LMNN)
\item Information Theoretic Metric Learning (ITML)
\item Sparse Determinant Metric Learning (SDML)
\item Least Squares Metric Learning (LSML)
\item Neighborhood Components Analysis (NCA)
\item Local Fisher Discriminant Analysis (LFDA)
\item Relative Components Analysis (RCA)
\end{enumerate}

\chapter{Experiments}

Datasets that we used for our experiments from 

https://archive.ics.uci.edu/ml/datasets/:

We chose the basic datasets that were mentioned in the original papers.

balance-scale, breast-cancer-wisconsin, ionosphere, iris, mice-protein, pima-indians-diabetes, sonar, soybean-large, wine

\chapter{Discussion}

Graphs, performace, caveats, what worked, what didnt work...

LMNN, ITML, SDML take too long to calculate

LMNN and LFDA perform very well

LMNN has too many hyper parameters

Simple Covariance metric was impossible to calculate on certain datasets

%\include{chap01}
%\include{chap02}
%\chapter{Mahalanobis metric}

\begin{align*}
  d(x,y)_{A} &= \sqrt{(x-y)^{T}A(x-y)} \\
         &= \sqrt{(x-y)^{T}L^{T}L(x-y)} \\
         &= \sqrt{(Lx-Ly)^{T}(Lx-Ly)} \\
         &= d(Lx, Ly)
\end{align*}
%\chapter*{Conclusion}
\addcontentsline{toc}{chapter}{Conclusion}

This worked, this didn't work. What we did.

\chapter{Introduction} \label{introduction}
\addcontentsline{toc}{chapter}{Introduction}

The need for measuring distance or similarity between data is ubiquitous in machine learning and creating custom metrics by hand is very difficult process. This has led to an introduction of distance metric learning where we learn distance metrics or similarities from data themselves.

One of the popular metrics is Mahalanobis distance, which generalizes Euclidean distance with a parameter matrix. We examine methods for learning this Mahalanobis distance and compare their performance on common datasets.

% Úvod do problematiky
% - Co už se udělalo
% - Co je známo
% - Co není známo
% - Co a pro Co a proč chceme naší prací objasnit naší prací objasnit
% - Jasná definice cíle práce
% - Struktura práce

% Related works
% - Co už udělali jiní
% - Co nového hodlá udělat autor

% v úvodu
% samostatná kapitola za úvodem samostatná kapitola za úvodem
% v závěru článku

\section*{Motivation}

\section*{Methodology}
\section*{Thesis structure}

%%%%%%%%%%%%%%%%%%%%%%%%%%%%%%%%%%%%%%%%%%%%%%%%%%%%%%%%%%%%%%%%%%%%%

\chapter{Notation}

More notation at http://www.deeplearningbook.org/contents/notation.html

\noindent Numbered equation:
\begin{equation*}\label{my favorite equation}
  e^{i\pi}=-1
\end{equation*}

\section{Math references} \label{mathrefs}
As mentioned in section \autoref{introduction}, different elements can 
be referenced within a document
 
\subsection{powers series} \label{subsection}
 
\begin{equation} \label{eq:1}
\sum_{i=0}^{\infty} a_i x^i
\end{equation}
 
The equation \ref{eq:1} is a typical power series.
 
In the subsection \ref{subsection} at the page \pageref{eq:1} an 
example of a power series was presented.

\begin{tabbing}
\hspace{200pt}\=\kill
$\mathbb{R}$ \> Set of real numbers \\
$\mathbb{R}^d$ \> Set of d-dimensional real-valued vectors \\
$\mathbb{R}^{c \times d}$ \> Set of c$\times$d real-valued matrices \\
$\mathbb{S}^{d \times d}_+$ \> Cone of symmetric PSD d$\times$d real-valued matrices \\
$\textbf{x}$ \> An arbitrary vector \\
$\textbf{\textit{M}}$ \> An arbitrary matrix \\
$\textbf{\textit{I}}$ \> Identity matrix \\
$\textbf{\textit{M}} \succeq 0$ \> PSD matrix M \\
$\parallel \cdot \parallel_p$ \> p-norm \\
$\parallel \cdot \parallel_\mathcal{F}$ \> Frobenius norm \\
$\parallel \cdot \parallel_*$ \> Nuclear norm \\
$\tr(\textbf{\textit{M}})$ \> Trace of matrix M \\
$[t]_+ = max(0, 1-t)$ \> Hinge loss function \\
$\xi$ \> Slack variable \\
$\Sigma$ \> Finite alphabet \\

$\mathcal{X} \subseteq \mathbb{R}^d$ \> Input (instance) space \\
$\mathcal{Y} = \{ 1, \ldots ,c \}$ \> Set of $c$ output labels (ground truth) \\
$\widehat{\mathcal{Y}} = \{ 1, \ldots ,\widehat{c} \} \subseteq \mathcal{Y}$ \> Set of $\widehat{c}$ known output labels \\
% $x$ \> String of finite size
$z=(x,y) \in \mathcal{X} \times \mathcal{Y}$ \> An arbitrary labeled instance \\
$\mathcal{L}=\{z_i=(x_i, y_i)\}^n_{i=1}$ \> Set of $n$ labeled training instances \\
$\mathcal{U} = \{x_i \}^m_{i=1}$ \> Set of $m$ unlabeled training instances \\
$\overline{z} = (x, \overline{y}) \in \mathcal{X} \times \widehat{\mathcal{Y}} $ \> An arbitrary instance with predicted label \\
$\overline{\mathcal{Z}} = \{\overline{z_i} = (x_i, \overline{y_i}) \}^m_{i=1}$ \> Set of $m$ training instances with predicted labels \\
$\mathcal{S} = \{ (x_i, x_j) \mid y_i = y_j \}$ \> Set of must-link constraints \\
$\mathcal{D} = \{ (x_i, x_j) \mid y_i \neq y_j \}$ \> Set of cannot-link constraints \\
$\mathcal{R} = \{ (x_i, x_j, x_k) \mid y_i = y_j \wedge y_i \neq y_k \}$ \> Set of relative constraints \\
\end{tabbing} 

%%%%%%%%%%%%%%%%%%%%%%%%%%%%%%%%%%%%%%%%%%%%%%%%%%%%%%%%%%%%%%%%%%%%%

\chapter{Mahalanobis metric}

\begin{align*}
  d(x,y)_{A} &= \sqrt{(x-y)^{T}A(x-y)} \\
         &= \sqrt{(x-y)^{T}L^{T}L(x-y)} \\
         &= \sqrt{(Lx-Ly)^{T}(Lx-Ly)} \\
         &= d(Lx, Ly)
\end{align*}

%%%%%%%%%%%%%%%%%%%%%%%%%%%%%%%%%%%%%%%%%%%%%%%%%%%%%%%%%%%%%%%%%%%%%

\chapter{SVD decomposition}

proof that $A^TA$ is PSD matrix

% http://www.deeplearningbook.org/contents/linear_algebra.html

%%%%%%%%%%%%%%%%%%%%%%%%%%%%%%%%%%%%%%%%%%%%%%%%%%%%%%%%%%%%%%%%%%%%%

\chapter{Methods}

In this work we chose to examine the following methods:

\begin{enumerate}
\item Simple Covariance Metric
\item Xing
\item Large Margin Nearest Neighbor (LMNN)
\item Information Theoretic Metric Learning (ITML)
\item Sparse Determinant Metric Learning (SDML)
\item Least Squares Metric Learning (LSML)
\item Neighborhood Components Analysis (NCA)
\item Local Fisher Discriminant Analysis (LFDA)
\item Relative Components Analysis (RCA)
\end{enumerate}

- Teoretická analýza

Zmínit lokální metriky, online, ...

%%%%%%%%%%%%%%%%%%%%%%%%%%%%%%%%%%%%%%%%%%%%%%%%%%%%%%%%%%%%%%%%%%%%%

\chapter{Experiments}

Datasets that we used for our experiments from 

https://archive.ics.uci.edu/ml/datasets/:

We chose the basic datasets that were mentioned in the original papers.

balance-scale, breast-cancer-wisconsin, ionosphere, iris, mice-protein, pima-indians-diabetes, sonar, soybean-large, wine

Avast dataset?

%%%%%%%%%%%%%%%%%%%%%%%%%%%%%%%%%%%%%%%%%%%%%%%%%%%%%%%%%%%%%%%%%%%%%

\chapter{Trivial toy dataset}

	*** **** *** *** *** * ** **** * * **** ***
	-  -   -    -    -   -   -    -    -     -


	********
	********

	--------
	--------


%%%%%%%%%%%%%%%%%%%%%%%%%%%%%%%%%%%%%%%%%%%%%%%%%%%%%%%%%%%%%%%%%%%%%

\chapter{Hierarchical Grid search}

how it works - scheme

how cross-validation is in memory

%%%%%%%%%%%%%%%%%%%%%%%%%%%%%%%%%%%%%%%%%%%%%%%%%%%%%%%%%%%%%%%%%%%%%

\chapter{Discussion}

Graphs, performace, caveats, what worked, what didnt work...

LMNN, ITML, SDML take too long to calculate

LMNN and LFDA perform very well

LMNN has too many hyper parameters

Simple Covariance metric was impossible to calculate on certain datasets

%%%%%%%%%%%%%%%%%%%%%%%%%%%%%%%%%%%%%%%%%%%%%%%%%%%%%%%%%%%%%%%%%%%%%

\chapter{Experiments}

Little intro.

\section{Datasets}

\section{Preprocessing data}

\section{Dimensionality reduction}

%%%%%%%%%%%%%%%%%%%%%%%%%%%%%%%%%%%%%%%%%%%%%%%%%%%%%%%%%%%%%%%%%%%%%

\chapter*{Conclusion}
\addcontentsline{toc}{chapter}{Conclusion}

This worked, this didn't work. What we did.

\chapter*{Recommendations for future work}

%%%%%%%%%%%%%%%%%%%%%%%%%%%%%%%%%%%%%%%%%%%%%%%%%%%%%%%%%%%%%%%%%%%%%

\chapter*{Test}
hello world \\ hello world2
$x=5$
$\frac{5}{6}$
$\dfrac{5}{6}$
\[111\]
$$121$$



An~example citation: \cite{Andel07}

\section{Title of the first subchapter of the first chapter}


\begin{figure}
    \centering
    \includegraphics[width=30mm]{img/logo-en}
    \caption{Logo of MFF UK}
    \label{fig:mff}
\end{figure}


% Example 1
\ldots when Einstein introduced his formula
\begin{equation}
e = m \cdot c^2 \; ,
\end{equation}
which is at the same time the most widely known
and the least well understood physical formula.
% Example 2
\ldots from which follows Kirchhoff’s current law:
\begin{equation}
\sum_{k=1}^{n} I_k = 0 \; .
\end{equation}
Kirchhoff’s voltage law can be derived \ldots
% Example 3
\ldots which has several advantages.
\begin{equation}
I_D = I_F - I_R
\end{equation}
is the core of a very different transistor model. \ldots

%%%%%%%%%%%%%%%%%%%%%%%%%%%%%%%%%%%%%%%%%%%%%%%%%%%%%%%%%%%%%%%%%%%%%

\section{Goals}

\subsection{Foundation stone}

Many researches devoted to learn a Mahalanobis distance metric, which has mostly been used to improve the performance of kNN classification. Most of the research has been focused on learning this metric from labeled training instances.

In this thesis, we review existing state-of-the-art methods for learning global Mahalanobis distance metric (Xing, LMNN, NCA, ...) and multiple local Mahalanobis distance metrics (PLML, ...) and compare them on several popular datasets. Learning the full Mahalanobis metric means learning $d^2$ parameters and so we also try to restrict the existing methods to learn only a diagonal of the Mahalanobis distance metric with $d$ parameters.

There was no unified library containing the algorithms for learning the distance metrics and so we created a library with unified interface according to the current standards.

\subsection{Nice to have}

In the next part of our work we apply the known methods from supervised setting to semi-supervised where we have both labeled and unlabeled training instances. The idea is to use the Mahalanobis metric distance learnt from labeled pairs for clustering the unlabeled instances. It is possible (and very likely) that the unlabeled instances contain more classes.

We examine this for methods using global metrics and also for methods using multiple local distance metrics where we needed to adapt existing clustering algorithms for our multi-metric setting.

\subsection{Moonshot}

In the last section of the thesis we examined an iterative distance metric learning process consisting of these steps:
\begin{enumerate*}
\item{learning a distance metric from the labeled data}
\item{applying a non-supervised clustering method to obtain new labels (and possibldifferent number of classes)}
\item{repeat}
\end{enumerate*}
.

\subsection{Not included}

online metrics? - usually global

%%%%%%%%%%%%%%%%%%%%%%%%%%%%%%%%%%%%%%%%%%%%%%%%%%%%%%%%%%%%%%%%%%%%%
%%%%%%%%%%%%%%%%%%%%%%%%%%%%%%%%%%%%%%%%%%%%%%%%%%%%%%%%%%%%%%%%%%%%%
%%%%%%%%%%%%%%%%%%%%%%%%%%%%%%%%%%%%%%%%%%%%%%%%%%%%%%%%%%%%%%%%%%%%%

%%% Bibliography
\include{bibliography}

%%% Figures used in the thesis (consider if this is needed)
\listoffigures

%%% Tables used in the thesis (consider if this is needed)
%%% In mathematical theses, it could be better to move the list of tables to the beginning of the thesis.
\listoftables

%%% Abbreviations used in the thesis, if any, including their explanation
%%% In mathematical theses, it could be better to move the list of abbreviations to the beginning of the thesis.
\chapwithtoc{List of Abbreviations}

%%% Attachments to the master thesis, if any. Each attachment must be
%%% referred to at least once from the text of the thesis. Attachments
%%% are numbered.
%%%
%%% The printed version should preferably contain attachments, which can be
%%% read (additional tables and charts, supplementary text, examples of
%%% program output, etc.). The electronic version is more suited for attachments
%%% which will likely be used in an electronic form rather than read (program
%%% source code, data files, interactive charts, etc.). Electronic attachments
%%% should be uploaded to SIS and optionally also included in the thesis on a~CD/DVD.
\chapwithtoc{Attachments}

\openright
\end{document}
